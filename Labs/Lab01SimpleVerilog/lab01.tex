\chapter{Introduction to Verilog}
\label{introductionToVerilog}
\graphicspath{ {./Lab01SimpleVerilog/Fig} }


\hypertarget{objective}{%
\section{\texorpdfstring{Objective }{Objective }}
\label{section:IntroVerilogObjective}}

The objective of this lab is to introduce you to the Quartus II
software, design entry using Verilog and circuit simulation.


\hypertarget{part-1-setting-up-a-project-in-quartus-and-running-a-testbench}{%
\section{Part 1: Setting up a project in Quartus and running a testbench}
\label{part-1-setting-up-a-project-in-quartus-and-running-a-testbench}}

\begin{enumerate}
\def\labelenumi{\arabic{enumi}.}
\item
  Select an appropriate working directory for your project. I would
  recommend selecting your network drive.

  \begin{enumerate}
  \def\labelenumii{\alph{enumii}.}
  \item
    Create a new folder \emph{lab1},
  \item
    Create another folder within \emph{lab1} called \emph{andgate2},
  \item
    Download \emph{andgate2.v} and \emph{andgate2\_tb.v} from Canvas,
  \item
    Save these files in andgate2 directory.
  \end{enumerate}
\item
  Start Quartus II 18.1 (64-bit).

  \begin{enumerate}
  \def\labelenumii{\alph{enumii}.}
  \item
    If you are prompted by a License Setup choose the free option. You
    may need to restart Quartus if this happens.
  \end{enumerate}
\item
  Select \emph{File -\textgreater{} New Project Wizard.}
\item
  In the \textbf{Directory, Name, Top-Level Entity} page of the New
  Project Wizard pop-up:

  \begin{enumerate}
  \def\labelenumii{\alph{enumii}.}
  \item
    To the right of the ``What is the working directory'' box click the
    \ldots{} button,
  \item
    In the Select Folder pop-up, navigate so you can see the andgate2
    directory created in step 1,
  \item
    Select the andgate2 folder, click Select Folder,
  \item
    In the ``What is the name of this project'' field type
    \emph{andgate2}
  \item
    click \emph{Next}.
  \end{enumerate}
\item
  In the \textbf{Project Type} page of the New Project Wizard pop-up:

  \begin{enumerate}
  \def\labelenumii{\alph{enumii}.}
  \item
    Select the \emph{Empty project} radio button,
  \item
    click \emph{Next}.
  \end{enumerate}
\item
  In the \textbf{Add Files} page of the New Project Wizard pop-up:

  \begin{enumerate}
  \def\labelenumii{\alph{enumii}.}
  \item
    Click the \ldots{} button to the right of File name,
  \item
    In the Select File pop-up, navigate to, and select,
    \emph{andgate2.v} and \emph{andgate2\_tb.v}, click Open,
  \item
    The file should appear in the window below,
  \item
    Click \emph{Next}
  \end{enumerate}
\item
  In the \textbf{Family \& Device Settings} page of the New Project
  Wizard pop-up:

  \begin{enumerate}
  \def\labelenumii{\alph{enumii}.}
  \item
    Device family, Family: Cyclone V
  \item
    Package: FBGA
  \item
    Pin Count: 672
  \item
    Speed Grade: 7\_H6
  \item
    Select Specific device selected in `Available devices' list
  \item
    From the list of available devices, select: 5CGXFC5C6F27C7
  \item
    Click Next
  \end{enumerate}
\item
  In the \textbf{EDA Tool Settings} page of the New Project Wizard
  pop-up:

  \begin{enumerate}
  \def\labelenumii{\alph{enumii}.}
  \item
    In the Simulation row

    \begin{enumerate}
    \def\labelenumiii{\roman{enumiii}.}
    \item
      Tool Name column: ModelSim-Altera
    \item
      Formats column: Verilog HDL
    \end{enumerate}
  \item
    Leave other defaults alone
  \item
    Click Next
  \end{enumerate}
\item
  In the \textbf{Summary} page of the New Project Wizard pop-up:

  \begin{enumerate}
  \def\labelenumii{\alph{enumii}.}
  \item
    Review information,
  \item
    Click Finish.
  \end{enumerate}
\item
  Back in the main Quartus II window, Click \emph{Tools -\textgreater{}
  Options..}.
\item
  In the Options pop-up:

  \begin{enumerate}
  \def\labelenumii{\alph{enumii}.}
  \item
    Select \emph{EDA Tool Options} from the Category menu,
  \item
    If the last row, ``ModelSim-Altera'' is blank, click on the \ldots{}
    button at right and navigate to the
    \emph{C:\textbackslash intelFPGA\_lite\textbackslash18.1\textbackslash modelsim\_ase\textbackslash{}},
    select the \emph{win32aloem} folder, the click Select Folder,
  \item
    Click Ok.
  \end{enumerate}
\item
  \includegraphics[width=3.9in,height=1.8in]{image1.png}Click
  on the Files tab in the \emph{Project Navigator} pane.
\item
  Right click on \emph{andgate2\_tb} in the \emph{Project Navigator}
  pane and select Set as Top-Level entity.
\item
  Double click on \emph{andgate2}.
\item
  If you added the Verilog file in the correct directory and included it
  in the project, a Verilog file should pop up on the right.
\item
  In the main Quartus II window, click on \emph{Processing
  -\textgreater{} Start -\textgreater{} Start Analysis \& Elaboration.}
  This may take some time, so be patient.
\item
  If you did everything correctly you should

  \begin{enumerate}
  \def\labelenumii{\alph{enumii}.}
  \item
    Notice that andgate\_tb is the new top-level entity in the Hierarchy
    pane. Expand the andgate2\_tb by clicking on the ``\textgreater''
    arrow to see the entities inside it.
  \item
    You should see the following messages in the console area, the
    bottom pane.
  \end{enumerate}
\end{enumerate}

\begin{quote}
\includegraphics{image2.png}
\end{quote}

\begin{enumerate}
\def\labelenumi{\arabic{enumi}.}
\setcounter{enumi}{17}
\item
  In the main Quartus II window, click \emph{Tools -\textgreater{} Run
  Simulation Tool -\textgreater{} RTL Simulation}. The ModelSim program
  will launch. This may take a few moments, be patient. If you get a
  pop-up Nativelink Error window, then go back and check and fix the
  directory in step 11.
\item
  In ModelSim, click Simulate -\textgreater{} Start Simulation
\item
  In the Start Simulation pop-up, expand the \emph{work} library by
  clicking on the ``+'' at left. click on \emph{andgate2\_tb} and click
  \emph{Ok}.
\item
  In the sim pane, right mouse click on uut and select \emph{Add Wave}.
\end{enumerate}

\begin{quote}
\includegraphics[width=4.10417in,height=1.75in]{image3.png}
\end{quote}

\begin{enumerate}
\def\labelenumi{\arabic{enumi}.}
\setcounter{enumi}{21}
\item
  Choose \emph{Simulate -\textgreater{} Run -\textgreater{} Run 100}.
  You should see inputs and output from andgate2. If you see only a
  small green portion of the waveform on the left margin of the timing
  diagram, you will need to zoom in on the waveform as follows. First
  click somewhere in the timing diagram (area under ``Undocking tool''
  in the image below) and then click on the ``Zoom all tool'' shown in
  following image.
\item
  \protect\hypertarget{Part_1_Step_23}{}{}Save this waveform as an image
  as follows:

  \begin{enumerate}
  \def\labelenumii{\alph{enumii}.}
  \item
    Undock the Wave pane by clicking the undocking tool icon.
  \end{enumerate}
\end{enumerate}

\begin{quote}
\includegraphics{image4.png}
\end{quote}

\begin{enumerate}
\def\labelenumi{\alph{enumi}.}
\setcounter{enumi}{1}
\item
  Resize the undocked Wave window vertically by grabbing its top edge
  and dragging down. Make the window tall enough to fit all the waves
  with a little room to spare.
\end{enumerate}

\begin{quote}
\includegraphics{image5.png}
\end{quote}

\begin{enumerate}
\def\labelenumi{\alph{enumi}.}
\setcounter{enumi}{2}
\item
  Click the Zoom all tool to file the available horizontal space with
  the waveform.
\item
  Click File -\textgreater{} Export -\textgreater{} Image
\end{enumerate}

\begin{quote}
If this does not work, you can take a screen shot of the window by
pressing Alt-Print Screen. The ``Alt'' captures the currently active
window into the graphics buffer.
\end{quote}

\begin{enumerate}
\def\labelenumi{\alph{enumi}.}
\setcounter{enumi}{4}
\item
  Navigate to your project directory, provide a File name, then click
  Save
\item
  Exit Modelsim using File -\textgreater{} Quit. Do not save wave
  commands.
\end{enumerate}

\begin{enumerate}
\def\labelenumi{\arabic{enumi}.}
\setcounter{enumi}{23}
\item
  Back in Quartus, close your current project using File -\textgreater{}
  Close Project. Save if needed.
\end{enumerate}

\hypertarget{part-2-symbolic-to-verilog-timing-diagram-truth-table.}{%
\section{Part 2: Symbolic to Verilog , Timing Diagram, Truth Table}
\label{part-2-symbolic-to-verilog-timing-diagram-truth-table.}}

Write Verilog code to realize the function \emph{f02 = a' + bc'} Note
that this symbolic expression is written using the notation used in
class. This is not a valid Verilog expression.

\begin{enumerate}
\def\labelenumi{\arabic{enumi}.}
\item
  Create \textbf{a new project} folder within your \emph{lab1} directory
  called \emph{function02.}
\item
  Download \emph{function02.v} and \emph{function02\_tb.v} from Canvas
  to the project directory.
\item
  Create a project for these two files using the steps above.
\item
  \protect\hypertarget{Part_2_Step_4}{}{}Modify the line of code that
  starts with \emph{assign} to realize the function \emph{f02} shown
  above.
\item
  Modify \emph{function02\_tb.v} so that \emph{f02} is run through every
  combination of inputs. Assert the inputs in increasing binary
  numbering order starting from 0,0,0 and going to 1,1,1.
\item
  Perform simulation using the given testbench as described in previous
  steps. You will need to ``run 100'' twice as the simulation is over
  100ns long.
\item
  \protect\hypertarget{Part_2_Step_7}{}{}Save this waveform as an image
  as done in the previous section. If the waveform is missing, you can
  add it back in using View -\textgreater{} Waveform.
\item
  \protect\hypertarget{Part_2_Step_8}{}{}From the information in the
  timing diagram, produce a truth table for \emph{f02}. Remember that a
  truth table is an enumeration of every possible input and the
  associated output. Please look at Chapter 2 in the textbook for some
  examples if you are unclear about how to setup a truth table.
\end{enumerate}


\hypertarget{part-3-verilog-to-symbolic-truth-table-circuit-diagram}{%
\section{Part 3: Verilog to Symbolic, Truth Table, Circuit Diagram}
\label{part-3-verilog-to-symbolic-truth-table-circuit-diagram}}

The Verilog code in the file function03 contains a complete circuit for
\emph{f03}. You will use the Quartus tools to get a timing diagram for
the function and, by looking at the Verilog code, determine the symbolic
form and circuit diagram.

\begin{enumerate}
\def\labelenumi{\arabic{enumi}.}
\item
  Create \textbf{a new project} folder within your \emph{lab1} directory
  called \emph{function03.}
\item
  Download \emph{function03.v} and \emph{function03}\_tb.v from Canvas
  to the project directory.
\item
  Create a project for these two files using the steps above.
\item
  Modify \emph{function03\_tb.v} so that \emph{f03} is run through every
  combination of inputs. Assert the inputs in increasing binary
  numbering order starting from 0,0,0 and going to 1,1,1.
\item
  Perform simulation using this test bench as described in previous
  steps. You will need to ``run 100'' twice as the simulation is over
  100ns long.
\item
  \protect\hypertarget{Part_3_Step_6}{}{}Save this waveform as an image,
  but with the following changes.

  \begin{enumerate}
  \def\labelenumii{\alph{enumii}.}
  \item
    Resize the area containing the names of the signals by grabbing the
    right vertical bar of the name area and moving it right.
  \item
    Re-order the waves so that f03 is lowest. Do this by grabbing the
    name ``/function03\_tb/uut/f03 and moving it below all the other
    signals.
  \item
    Color the intermediate signals (p1, p2, p4, p7) yellow by right
    clicking on them, selecting properties. In the View tab of the Wave
    Properties pop-up, click the Colors\ldots{} button for Wave Color
    and choose Yellow, click Close, then OK.
  \item
    Change the color of \emph{f03} to red.
  \end{enumerate}
\item
  \protect\hypertarget{Part_3_Step_7}{}{}From the information in the
  timing diagram, produce a truth table.
\item
  \protect\hypertarget{Part_3_Step_8}{}{}From the information in
  \emph{function03.v} draw the circuit diagram for \emph{f03}.
\item
  \protect\hypertarget{Part_3_Step_9}{}{}From the information in
  \emph{function03.v} write down the symbolic form for \emph{f03}.
\end{enumerate}

\hypertarget{part-4-circuit-diagram-to-verilog-symbolic-truth-table}{%
\section{Part 4: Circuit Diagram to Verilog, Symbolic, Truth Table}
\label{part-4-circuit-diagram-to-verilog-symbolic-truth-table}}

Write Verilog code to realize the function \emph{f04} shown in the
circuit diagram below.

\includegraphics{image6.png}

\begin{enumerate}
\def\labelenumi{\arabic{enumi}.}
\item
  Create \textbf{a new project} folder within your \emph{lab1} directory
  called \emph{function04.}
\item
  Download \emph{function04.v} and \emph{function04}\_tb.v from Canvas
  to the project directory.
\item
  Create a project for these two files using the steps above.
\item
  \protect\hypertarget{Part_4_Step_4}{}{}Modify \emph{function04.v} by
  writing an assignment statement for each of \emph{o1}, \emph{a1},
  \emph{a2}, and \emph{f04}.
\item
  Modify \emph{function04\_tb.v} so that \emph{f04} is run through every
  combination of inputs. Assert the inputs in increasing binary
  numbering order starting from 0,0,0 and going to 1,1,1.
\item
  Perform simulation using this test bench as described in previous
  steps. You will need to ``run 100'' twice as the simulation is over
  100ns long.
\item
  \protect\hypertarget{Part_4_Step_7}{}{}Save this waveform as an image
  as done in a previous section. Color intermediate signals (01, a1, a2)
  yellow and output red.
\item
  \protect\hypertarget{Part_4_Step_8}{}{}From the information in the
  timing diagram, produce a truth table.
\end{enumerate}

\hypertarget{turn-in}{%
\section{Turn in:}
\label{turn-in}}

Make a record of your response to numbered items below and turn them in
a single copy as your team's solution on Canvas using the instructions
posted there. Include the names of both team members at the top of your
solutions. Use complete English sentences to introduce what each of the
following listed items (below) is and how it was derived.

\textbf{Part 1:} \protect\hyperlink{Part_1_Step_23}{Step 23} Timing
diagram of AND gate

\textbf{Part 2:} \protect\hyperlink{Part_2_Step_4}{Step 4} Verilog code
for \emph{f02}

\textbf{Part 2:} \protect\hyperlink{Part_2_Step_7}{Step 7} Timing
diagram of \emph{f02}

\textbf{Part 2:} \protect\hyperlink{Part_2_Step_8}{Step 8} Truth table
of \emph{f02}

\textbf{Part 3:} \protect\hyperlink{Part_3_Step_6}{Step 6} Timing
diagram of \emph{f03}

\textbf{Part 3:} \protect\hyperlink{Part_3_Step_7}{Step 7} Truth table
of \emph{f03}

\textbf{Part 3:} \protect\hyperlink{Part_3_Step_8}{Step 8} Circuit
Diagram of \emph{f03}

\textbf{Part 3:} \protect\hyperlink{Part_3_Step_9}{Step 9} Symbolic form
of \emph{f03}

\textbf{Part 4:} \protect\hyperlink{Part_4_Step_4}{Step 4} Just the 4
Verilog assign statement for \emph{o1}, \emph{a1}, \emph{a2}, and
\emph{f04}.

\textbf{Part 4:} \protect\hyperlink{Part_4_Step_7}{Step 7} Timing
diagram of \emph{f04}

\textbf{Part 4:} \protect\hyperlink{Part_4_Step_8}{Step 8} Truth table
of \emph{f04}


