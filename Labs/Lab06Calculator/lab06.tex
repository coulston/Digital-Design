\chapter{Calculator With Friendly Output}
\label{chapter:calc}
\graphicspath{{./Lab06Calculator/Fig}}

\section{Outcomes and Objectives}

The outcome of this lab is to instantiate a calculator with signed
decimal output making it easy for anyone to use the circuit.
Through this process you will achieve the following
learning objectives.
\begin{itemize}
	\item \Paste{bok:CC_WireLogic}
	\item \Paste{bok:CC_Glue_Combo}
	\item \Paste{bok:CC_Combos}
	\item \Paste{VER:AlwaysCaseZ}
	\item\Paste{HDL:Synthesis}
\end{itemize}



\section{Calculator with Friendly Output}

This week you are going to build a calculator that can add or
subtract 4-bit values using the input and output shown in 
Figure~\ref{fig:calcDevBoard} and display the results as
decimal, base-10, values, not as hexadecimal values.

\begin{figure}[ht]
\includegraphics{ image1.png}
\caption{The input and output of the calculator digital circuit.}
\label{fig:calcDevBoard}
\end{figure}

On the surface, this should require nothing more
than connecting some slide switches to the x and y inputs of an
adder/subtractor which sends its output to a 7-segment display. And for
the most part this is correct. However, instead of displaying the input
and output of the adder as hexadecimal values, you will display them as
2-digit decimal values. 

The user input and output are shown in Figure~\ref{fig:calcDevBoard}.
The user enters a pair of 4-bit operands using the left-most slide
switches, \device{xSlide} and \device{ySlide}. The value entered for
\device{xSlide} is displayed on the two (red) \device{xDisplay}
7-segment displays. The value entered for \device{ySlide} is displayed
on the two (green) \device{yDisplay} 7-segment displays. The leftmost
the \device{addSub} buttons specify the operation performed on
\device{xSlide} and \device{ySlide}. The result is \device{xSlide} +
\device{ySlide} or \device{xSlide} - \device{ySlide}. 



The \device{interp} button determines how the values are displayed on the
7-segment display. When unpressed, the 7-segment displays show the
decimal value, when pressed, the 7-segment displays show 2's complement.
This will be explained in the next section. As we have only four 7-segment displays on
board, the same 7-segment displays of operand Y will be used to show the operation
result (yellow) when the \device{yOrResult} button is pressed.

\section{System Architecture}

The system architecture shown in Figure~\ref{fig:sysArchCalc} shows the adder subtractor
processing the \device{xSlide} and \device{ySlide} inputs. The 4-bit x, y and result
values are processed by the \hdl{sigUnsig} module before being displayed on the
7-segment displays. It is now time to turn our attention to this module.

\begin{figure}[ht]
\includegraphics[width=0.5\paperwidth]{ image2.png}
\caption{The system architecture of the calculator.}
\label{fig:sysArchCalc}
\end{figure}


\section{Module: sigUnsign}

The significant design problem in today's lab comes in this section,
building the \hdl{sigUnsign} module that shows up three times in Figure~\ref{fig:sysArchCalc}. 
This module takes in a 4-bit value and
displays a 2-digit signed or unsigned representation on a pair of
7-segment displays. The \hdl{sigUnsign} module declaration is shown in Listing~\ref{listing:cacSigUnsigModule}.


\begin{lstlisting}[language=Verilog,
 caption={Module declaration for the sigUnsig module.},
 label={listing:cacSigUnsigModule},
 frame=single]
 module sigUnsig(x, interp, ovf, msDisplay, lsDisplay);    
    input  wire [3:0]  x;	 	 
    input  wire        interp;
    input  wire        ovf;
    output wire [6:0]  msDisplay, lsDisplay;
 \end{lstlisting}

The 4-bit input \hdl{x} is interpreted as either signed (2's complement
value) when \hdl{interp = 1} or unsigned (regular binary number) when
\hdl{interp = 0}.  The \device{msDisplay} is the most significant (ms)
symbol being displayed and \device{lsDisplay} is the least significant (ls)
symbol being display.  The term ``symbol'' is used because more than
one type of information can be displayed depending on the values
of the inputs.  Let's explore this.

For example, let \hdl{x = 4'b1100}. 
\begin{itemize}
\item 
If \hdl{interp = 1'b0} then
\hdl{x} is interpreted as unsigned and its value is 12. Then
the \device{msDisplay} should show ``1'' and \device{lsDisplay} ``2''. 

\item 
If \hdl{interp = 1'b1} then
\hdl{x} is interpreted as 2's complement and its value is -4. Then
the \device{msDisplay} should show ``-'' and \device{lsDisplay} ``4''. 

\item 
In the previous two cases we assumed, without stating it, that \hdl{ovf = 1'b0}.
If \hdl{ovf = 1'b1} then the operation which generated \hdl{x} overflowed
and the value of \hdl{x} is invalid.  In this case both display's should show ``X''.
Since we are working with 7-segments, our ``X'' looks much more like ``H'' :(
\end{itemize}

Not complete Table~\ref{table:calcSigUnsign} by filling in the values of \hdl{msDisplay} and
\hdl{lsDisplay} for a signed and unsigned interpretation, assuming
\emph{ovf}=0. If the interpreted value is
positive and a single digit then assign \emph{msDisplay} blank. If the
interpreted value is negative then assign \emph{msDisplay} ``-``. If the
interpreted value is greater than 10, assign \emph{msDisplay} ``1''.


\begin{longtable}[]{@{}
|  >{\raggedright\arraybackslash}p{(\columnwidth - 8\tabcolsep) * \real{0.1999}}|
  >{\raggedright\arraybackslash}p{(\columnwidth - 8\tabcolsep) * \real{0.2000}}|
  >{\raggedright\arraybackslash}p{(\columnwidth - 8\tabcolsep) * \real{0.2000}}|
  >{\raggedright\arraybackslash}p{(\columnwidth - 8\tabcolsep) * \real{0.2000}}|
  >{\raggedright\arraybackslash}p{(\columnwidth - 8\tabcolsep) * \real{0.2000}}|@{}}
\caption{The output of the \hdl{sigUnsig} module when \hdl{ovf=0}.}\label{table:calcSigUnsign}\tabularnewline
\toprule()
\multirow{2}{*}{4-bit input x} &
\multicolumn{2}{>{\raggedright\arraybackslash}p{(\columnwidth - 8\tabcolsep) * \real{0.4000} + 2\tabcolsep}}{%
\hdl{interp = 0} Unsigned } 
 &
\multicolumn{2}{|>{\raggedright\arraybackslash}p{(\columnwidth - 8\tabcolsep) * \real{0.4000} + 2\tabcolsep}|@{}}{%
\hdl{interp = 1} Signed } \\ \cline{2-5}

&\hdl{msDisplay} & \hdl{lsDisplay} & \hdl{msDisplay} & \hdl{lsDisplay} \\  

\midrule()
\endfirsthead
\toprule()
\multirow{2}{*}{4-bit input x} &
\multicolumn{2}{>{\raggedright\arraybackslash}p{(\columnwidth - 8\tabcolsep) * \real{0.4000} + 2\tabcolsep}}{%
\hdl{interp = 0} Unsigned } 
 &
\multicolumn{2}{|>{\raggedright\arraybackslash}p{(\columnwidth - 8\tabcolsep) * \real{0.4000} + 2\tabcolsep}|@{}}{%
\hdl{interp = 1} Signed } \\  \cline{2-5}

&\hdl{msDisplay} & \hdl{lsDisplay} & \hdl{msDisplay} & \hdl{lsDisplay} \\  
\midrule()
\endhead
4'b0000 & blank & 0 & blank & 0 \\ \hline
4'b0001 & & & & \\ \hline
4'b0010 & & & & \\ \hline
4'b0011 & & & & \\ \hline
4'b0100 & & & & \\ \hline
4'b0101 & & & & \\ \hline
4'b0110 & & & & \\ \hline
4'b0111 & & & & \\ \hline
4'b1000 & & & & \\ \hline
4'b1001 & & & & \\ \hline
4'b1010 & & & & \\ \hline
4'b1011 & & & & \\ \hline
4'b1100 & 1 & 2 & - & 4 \\ \hline
4'b1101 & & & & \\ \hline
4'b1110 & & & & \\ \hline
4'b1111 & & & & \\
\bottomrule()
\end{longtable}


Take a moment and look at the patterns in Table~\ref{table:calcSigUnsign}.  You should make the 
following important observations.
\begin{itemize}

	\item \hdl{msDisplay} is assigned one of four values
	\begin{itemize}
		\item  \hdl{blank} when the interpretation of \hdl{x} is an unsigned or signed value less than  10.
		\item \hdl{1} when the interpretation of \hdl{x} is an unsigned value greater than 10.
		\item \hdl{-} when the interpretation of \hdl{x} is a \textbf{signed value less than 0}.
		\item \hdl{X} (the invalid character)when the \hdl{ovf = 1}.
	\end{itemize}

	\item \hdl{lsDisplay} is assigned one of four values, 
	\begin{itemize}
		\item \hdl{x} (the value of the \hdl{x} input) when the interpretation of \hdl{x} is a unsigned or signed value less than 10.
		\item \hdl{x-10} when the interpretation of \hdl{x} is an unsigned value greater than 10.
		\item \hdl{0-x} when the interpretation of \hdl{x} is a \textbf{signed value less than 0}.
		\item \hdl{X} (the invalid character) when the \hdl{ovf = 1}.
	\end{itemize}
\end{itemize}


\subsubsection{Why are we taking the 2's complement of \hdl{x}?}

Please take a moment and reflect on pair of rows where you are asked to interpret \hdl{x} as a \textbf{signed value less than 0}.
Under a signed (2's complement) interpretation, if the most significant bit of \hdl{x} is 1 then the value of \hdl{x} is less than 0.
In this case the \device{msDisplay}  7-segment display should be illuminated with a ``-'' to indicate negative.  The \device{lsDisplay} needs
to show the negation of \hdl{x} because the negation of a negative number is a positive number and we can use a \hdl{hexToSevenSeg}
module to display positive numbers.  This is a complex but important observation.

If you follow the above reasoning, there is a need to form the 2's complement of \hdl{x} in certain input situations.  You will form the
negation of \hdl{x} by subtracting \hdl{x} from 0, that is compute \hdl{0-x}.  You will do this by putting \hdl{4'b0000} on the \hdl{x} input
of an \hdl{addSub}, put the sigUnsign input \hdl{x} on the \hdl{y} input of an \hdl{addSub}, and hardwire the \hdl{fnc} input to \hdl{1'b1} so that the
\hdl{addSub} is hardwired to always subtract.


Formalized the observations in a more algorithmic syntax by completing Listing~\ref{listing:calcDisplayLogic}.
Do this by filling the values of \hdl{msDisplay} and \hdl{lsDisplay} for the different input conditions.  The 
values for these two signals are given in the two lists above.  Note that this code is NOT to be used in your 
actual code for this lab.

\pagebreak

\begin{lstlisting}[language=Verilog,
 caption={Logic that determines the output of the 4:1 muxes in Figure~\ref{fig:calcSigUnSigArch}.},
 label={listing:calcDisplayLogic},
 frame=single]
if        ( (interp == 0) && (x < 10) ) {    // y0 input
    msDisplay = 		lsDisplay = 	

} else if ( (interp == 0) && (x >= 10) ) {    // y1 input
    msDisplay =		lsDisplay =

} else if ( (interp == 1) && (x >= 0) ) {    // y0 input
    msDisplay =		lsDisplay = 

} else if ( (interp == 1) && (x < 0) ) {    // y2 input
    msDisplay = 		lsDisplay = 

}
\end{lstlisting}

Now that you know what should be displayed on  \device{msDisplay} 
and \device{lsDisplay}, let's look at how we an form these symbols on the
7-segment displays.  In order to do this, you need Figure~\ref{fig:calcSevenSeg},
the bit-order of the segments controlling the illumination o the segments. Remember
that the segments are active low, meaning a logic 0 illuminates a
segment. Thus, the 7-bit code 7'b0100100 illuminates the pattern ``2''.

\begin{figure}[ht]
\includegraphics[width=0.3\paperwidth]{ image3.png}
\caption{The logical arrangements of the segments in a 7-segment display.}
\label{fig:calcSevenSeg}
\end{figure}

Test your understanding o the \hdl{signUnsign} output by completing
Table~\ref{table:calcSevenSeg}. Do this by coloring in the segments of the 
7-segment displays that are illuminated for each of the inputs. Then write 
the binary and hexadecimal value to illuminate those patterns to the 
right of \hdl{msDisplay=7'b} and \hdl{lsDisplay=7'b}. 

\pagebreak

\begin{longtable}[]{@{}
|  >{\raggedright\arraybackslash}p{(\columnwidth - 8\tabcolsep) * \real{0.1458}}|
  >{\raggedright\arraybackslash}p{(\columnwidth - 8\tabcolsep) * \real{0.3170}}|
  >{\raggedright\arraybackslash}p{(\columnwidth - 8\tabcolsep) * \real{0.02}}|
  >{\raggedright\arraybackslash}p{(\columnwidth - 8\tabcolsep) * \real{0.1458}}|
  >{\raggedright\arraybackslash}p{(\columnwidth - 8\tabcolsep) * \real{0.3170}}|@{}}
\caption{For each set of inputs to the signUnsig module, determine the 7-segment display pattern.}\label{table:calcSevenSeg} \tabularnewline
\toprule()
Input & 7-segment pattern & & Input & 7-segment pattern \\
\midrule()
\endfirsthead
\toprule()
Input & 7-segment pattern & & Input & 7-segment pattern \\
\midrule()
\endhead
4'b0010

interp = 1

ovf = 0 &

\vspace{0.1cm}
\includegraphics{ image4.png}
\vspace{0.1cm}

msDisplay = 7'b

lsDisplay = 7'b & & 4'b0111

interp = 0

ovf = 0 &

\vspace{0.1cm}
\includegraphics{ image4.png}
\vspace{0.1cm}

msDisplay = 7'b

lsDisplay =7'b \\ \hline
4'b1100

interp = 0

ovf = 0 &

\vspace{0.1cm}
\includegraphics{ image5.png}
\vspace{0.1cm}

msDisplay = 7'b1111001 = 7'h79

lsDisplay =7'b0100100 = 7'h24 & & 4'b1000

interp = 1

ovf = 0 &

\vspace{0.1cm}
\includegraphics{ image4.png}
\vspace{0.1cm}

msDisplay = 7'b

lsDisplay =7'b \\ \hline
4'b1100

interp = 1

ovf = 0 &

\vspace{0.1cm}
\includegraphics{ image4.png}
\vspace{0.1cm}

msDisplay = 7'b

lsDisplay =7'b & & 4'b1010

interp = 1

ovf = 1 &

\vspace{0.1cm}
\includegraphics{ image4.png}
\vspace{0.1cm}

msDisplay = 7'b

lsDisplay = 7'b \\ \hline
\bottomrule()
\end{longtable}

Now we are ready to put the pieces of the sigUnsig module together. The
building blocks in Figure~\ref{fig:calcSigUnSigArch} are captured in the organization described
by Listing~\ref{listing:calcDisplayLogic}, along with some extra hardware.

\begin{figure}[ht]
\includegraphics[width=0.5\paperwidth]{ image6.png}
\caption{The internal architecture of the signUnsig module.}
\label{fig:calcSigUnSigArch}
\end{figure}

Complete Figure~\ref{fig:calcSigUnSigArch} by adding the following:

\begin{itemize}
\item
Connect the inputs of the 4:1 mux  using the logic described in Listing~\ref{listing:calcDisplayLogic}. 
\item
Connect the inputs of the and adder subtractors using the logic described in Listing~\ref{listing:calcDisplayLogic}. 
\item
  Wire the inputs of the comparator to generate the signal \hdl{xGE10} which is 
  logic 1 when \hdl{x} is greater than or equal to 10.
\item
  Wire the input of the rightmost hexToSevenSeg .
\end{itemize}

Al that remains is to define the contents of the \hdl{glueLogic} box in Figure~\ref{fig:calcSigUnSigArch}. 

\subsubsection{\hdl{glueLogic} always/casez statement}

the \hdl{glueLogic} box chooses which input of the 4 mux inputs to
route to the output. This is the logic that you formalized in 
Listing~\ref{listing:calcDisplayLogic}.  Note the signal \hdl{sign} which equals 1 when \hdl{x}
 represents a negative value when interpreted as a signed value.
 
 Now complete the truth table in Table~\ref{table:calcGlueLogic} for the \hdl{glueLogic} box
in Figure~\ref{fig:calcSigUnSigArch}. 


\begin{longtable}[]{@{}
|  >{\raggedright\arraybackslash}p{(\columnwidth - 8\tabcolsep) * \real{0.1999}}|
  >{\raggedright\arraybackslash}p{(\columnwidth - 8\tabcolsep) * \real{0.2000}}|
  >{\raggedright\arraybackslash}p{(\columnwidth - 8\tabcolsep) * \real{0.2000}}|
  >{\raggedright\arraybackslash}p{(\columnwidth - 8\tabcolsep) * \real{0.2000}}|
  >{\raggedright\arraybackslash}p{(\columnwidth - 8\tabcolsep) * \real{0.2000}}|@{}}
\caption{Truth table for the glueLogic box.}\label{table:calcGlueLogic}\tabularnewline
\toprule()
\begin{minipage}[b]{\linewidth}\raggedright
\hdl{ovf}
\end{minipage} & \begin{minipage}[b]{\linewidth}\raggedright
\hdl{interp}
\end{minipage} & \begin{minipage}[b]{\linewidth}\raggedright
\hdl{sign}
\end{minipage} & \begin{minipage}[b]{\linewidth}\raggedright
\hdl{xGE10}
\end{minipage} & \begin{minipage}[b]{\linewidth}\raggedright
\hdl{digSel}
\end{minipage} \\
\midrule()
\endfirsthead
\toprule()
\begin{minipage}[b]{\linewidth}\raggedright
\hdl{ovf}
\end{minipage} & \begin{minipage}[b]{\linewidth}\raggedright
\hdl{interp}
\end{minipage} & \begin{minipage}[b]{\linewidth}\raggedright
\hdl{sign}
\end{minipage} & \begin{minipage}[b]{\linewidth}\raggedright
\hdl{xGE10}
\end{minipage} & \begin{minipage}[b]{\linewidth}\raggedright
\hdl{digSel}
\end{minipage} \\
\midrule()
\endhead
1 & x & x & x & \\ \hline
0 & 0 & x & 0 & \\ \hline
0 & 0 & x & 1 & \\ \hline
0 & 1 & 0 & x & \\ \hline
0 & 1 & 1 & x & \\
\bottomrule()
\end{longtable}

It would make sense to use an always case statement to realize the logic
in Listing~\ref{listing:calcDisplayLogic}. However, an always case statement requires each of the 16
difference cases to be explicitly enumerated. However, the truth table
in Listing~\ref{listing:calcDisplayLogic} is most efficiently described using don't cares in the
input. Fortunately, the always/casez variation (note the ``z'' at the
end of ``case'') allows don't cares in the input in the form of ``?''.
For example, for the second row in Listing~\ref{listing:calcDisplayLogic}, the \{ovf, interp, sign,
xGE10\} vector has don't cares for the \emph{sign} value. Therefore, the
case for this row is 4'b01?0. \uline{It is imperative that you include a
``default'' case whenever you use a always/case statement.} This
combination of cases is shown in Listing~\ref{listing:cacAlwaysStaement}.


\begin{lstlisting}[language=Verilog,
 caption={The always/casez statement allows don't cares in the input.},
 label={listing:cacAlwaysStaement},
 frame=single]
 always @(*)
    casez ({ovf, interp, xGE10, x[3]})
        4'b01?0: digSel = 2'b00;
        default: digSel = 2'b11;
    endcase

 \end{lstlisting}

\subsubsection{\hdl{sigUnsig} Verilog code}
\protect\hypertarget{sigUnsign_Verilog}{}{}
The Verilog code for the
signUnsig module consists of 8 instantiation statements and an
always/casez statement. For this module, I want you to:

\begin{itemize}
\item
  Use the module declaration given in Listing~\ref{listing:cacSigUnsigModule}.
\item
  Use the module definitions for

  \begin{itemize}
  \item
    \hdl{genericMux4x1} posted on this lab's Canvas folder
  \item
    \hdl{sevenSegment} created in lab 02
  \item
    \hdl{genericAdderSubtractor} posted on a previous lab's Canvas folder
  \item
    \hdl{genericComparator} posted on a previous lab's Canvas folder
  \end{itemize}
\item
  Use localparm to give names to the 7-bit constant patterns (fill in
  the values for x).

  \begin{itemize}
  \item
   \hdl{ localparam {[}6:0{]} displayBlank = 7'bxxxxxxx;}
  \item
   \hdl{ localparam {[}6:0{]} displayOne = 7'bxxxxxxx;}
  \item
   \hdl{ localparam {[}6:0{]} displayMinus = 7'bxxxxxxx;}
  \item
   \hdl{ localparam {[}6:0{]} displayX = 7'bxxxxxxx;}
  \end{itemize}
\item
  Provide meaningful names to the wires in the module.
\item
  Properly tab-indent your code

  \begin{itemize}
  \item
    Single level for wire declarations
  \item
    Single level for component instantiations
  \item
    Two levels for casez statement
  \item
    Three levels for casez values
  \end{itemize}
\end{itemize}

\section{Testbench}
\label{section:calcTestbench}
  Run the testbench for the sigUnsig module provided on Canvas. Produce
  a timing diagram with the following characteristics. Zoom to fill the
  available horizontal space with the waveform. Color inputs green and
  outputs red. Order the traces from top to bottom as

  \begin{tabular}{p{3cm}p{3cm}p{3cm}}
  signal			& radix				& Color for trace \\ \hline
    x radix 		&	unsigned 		& Green  \\
    xMinus10 		&	 unsigned 	& Green  \\
    x 				&  decimal 		& Cyan  \\
    negativeX 		&  decimal 		& Cyan  \\
    interp 			& default 			& Green  \\
    ovf 			& default 			& Green  \\
    digSel 			&  unsigned 		& Yellow  \\
    xGE10 		& default 			& Yellow  \\
    msDisplay 		&  hex 			& Red  \\
    lsDisplay 		&  hex 			& Red  \\
  \end{tabular}

I do not want the signals from the testbench, but rather the signals
from inside the \hdl{sigUnsig} module. You can do this in \hdl{sigUnsig} by
expanding the \hdl{sigUnsig\_tb} instance in the left ModelSim pane and
selecting ``uut''. Since uut is an instance of the \hdl{sigUnsig} module, all
the signals accessible in the \hdl{sigUnsig} module are shown in the center
Object. You can add duplicates of signals by repeating the drag-and-drop
operation.

Your completed timing diagram should look something like the following.

\includegraphics{ image7.png}


\section{Pin-Assignment and Synthesis}

Use the image of the FPGA Development Board in Figure~\ref{fig:calcDevBoard} 
and the information in the C5G User Guide to determine the FPGA pins associated 
with the input and output devices used by the devices used by the \hdl{calculator} 
module.


\begin{longtable}[]{@{}
|  >{\raggedright\arraybackslash}p{(\columnwidth - 8\tabcolsep) * \real{0.1436}}|
  >{\raggedright\arraybackslash}p{(\columnwidth - 8\tabcolsep) * \real{0.1829}}|
  >{\raggedright\arraybackslash}p{(\columnwidth - 8\tabcolsep) * \real{0.1665}}|
  >{\raggedright\arraybackslash}p{(\columnwidth - 8\tabcolsep) * \real{0.2621}}|
  >{\raggedright\arraybackslash}p{(\columnwidth - 8\tabcolsep) * \real{0.2448}}|@{}}
  \caption{Pin Assignment for the calculator.}\label{table:calcPinAssignment}\tabularnewline
\toprule()
\begin{minipage}[b]{\linewidth}\raggedright
Segment
\end{minipage} & \begin{minipage}[b]{\linewidth}\raggedright
msXdisplay
\end{minipage} & \begin{minipage}[b]{\linewidth}\raggedright
lsXdisplay
\end{minipage} & \begin{minipage}[b]{\linewidth}\raggedright
msYorRESdisplay
\end{minipage} & \begin{minipage}[b]{\linewidth}\raggedright
lsYorRESdisplay
\end{minipage} \\
\midrule()
\endhead
seg{[}6{]} &AC22 & & & \\ \hline
seg{[}5{]} & & W21 & & \\ \hline
seg{[}4{]} & & & AE25& \\ \hline
seg{[}3{]} & & & & W18\\ \hline
seg{[}2{]} & & & & \\ \hline
seg{[}1{]} & & & & \\ \hline
seg{[}0{]} & & & & \\
\bottomrule()
\end{longtable}

\begin{longtable}[]{@{}
|  >{\raggedright\arraybackslash}p{(\columnwidth - 4\tabcolsep) * \real{0.3397}}|
  >{\raggedright\arraybackslash}p{(\columnwidth - 4\tabcolsep) * \real{0.3278}}|
  >{\raggedright\arraybackslash}p{(\columnwidth - 4\tabcolsep) * \real{0.3325}}|@{}}
\toprule()
\begin{minipage}[b]{\linewidth}\raggedright
\end{minipage} & \begin{minipage}[b]{\linewidth}\raggedright
x
\end{minipage} & \begin{minipage}[b]{\linewidth}\raggedright
y
\end{minipage} \\
\midrule()
\endhead
slide{[}3{]} & AE19& \\ \hline
slide{[}2{]} & & W11\\ \hline
slide{[}1{]} & & \\ \hline
slide{[}0{]} & & \\
\bottomrule()
\end{longtable}

\begin{longtable}[]{@{}
|  >{\raggedright\arraybackslash}p{(\columnwidth - 4\tabcolsep) * \real{0.3334}}|
  >{\raggedright\arraybackslash}p{(\columnwidth - 4\tabcolsep) * \real{0.3334}}|
  >{\raggedright\arraybackslash}p{(\columnwidth - 4\tabcolsep) * \real{0.3333}}|@{}}
\toprule()
YorRES & Key{[}1{]} & P12 \\
\midrule()
\endhead
interp & Key{[}2{]} & \\ \hline
addSub & Key{[}3{]} & \\ \hline
\bottomrule()
\end{longtable}

Complete the pin-assignment in Quartus, compile your design and download to the FGPA
development boards. Once you get your design working, demonstrate it to a member of the lab team.


\section{Turn in}

You may work in teams of at most two. Make a record of your response to
the items below and turn them in a single copy as your team's solution
on Canvas using the instructions posted there. Include the names of both
team members at the top of your solutions. Use complete English
sentences to introduce what each of the following listed items (below)
is and how it was derived. In addition to this submission, you will be
expected to demonstrate your circuit at the beginning of your lab
section next week.

\subsubsection{signUnsig Module}

\begin{itemize}
\item
  Complete Table~\ref{table:calcSigUnsign}.
\item
  Complete Table~\ref{table:calcSevenSeg}.
\item
  Complete the code in Listing~\ref{listing:calcDisplayLogic}.
\item
  Complete Figure~\ref{fig:calcSigUnSigArch}, including:

  \begin{itemize}
  \item
    Constant values on inputs of 4:1 mux
  \item
    Constant value on the input of the right-most hexToSeventSeg
  \item
    Value on the input of the adder subtractors
  \item
    Values on the input of the comparator
  \end{itemize}
\item
  Complete Table~\ref{table:calcSevenSeg}.
\item
  Complete Table~\ref{table:calcGlueLogic}.
\item
  \protect\hyperlink{sigUnsign_Verilog}{Verilog code for the body of the
  sigUnsig module} (courier 8-point font single spaced), leave out
  header comments.
\end{itemize}

\subsubsection{Testbench}
\begin{itemize}
\item Complete testbench and timing diagram from Section~\ref{section:calcTestbench}.
\end{itemize}

\subsubsection{Pin-Assignment and Synthesis}
\begin{itemize}
\item Complete the pin assignment in table~\ref{table:calcPinAssignment}.
\end{itemize}

\section{Bonus: Ovf Logic}

The default configuration of the system architecture ignores any
overflow generated by the adder subtractor. If you choose, you may
implement the logic necessary to determine if overflow occurs in the
selected interpretation. In order to receive credit, your circuit needs
to work under \uline{all combination} of addSub and interp. Overflow for
unsigned subtraction will require some careful analysis.

Your solution should have 2 LEDs, one for signed and one for unsigned.
The unsigned overflow LED should illuminate when overflow will occur if
the numbers are interpreted as unsigned numbers. The signed overflow LED
is on when an overflow will occur if the numbers are interpreted as
two's complement numbers.

For example, if the x and y inputs are 1001 and the operation is
addition, then both signed and unsigned LEDs will illuminate.