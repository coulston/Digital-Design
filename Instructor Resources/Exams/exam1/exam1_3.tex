\documentclass{article}
\usepackage{epsfig, latexsym}

\begin{document}

\newcommand{\SOPmin}{${\rm SOP}_{\rm min} \ $}
\newcommand{\POSmin}{${\rm POS}_{\rm min} \ $}
\newcommand{\bs}{\backslash}


\title{
\Huge{CSE 271 -- Spring 2002}\\
\normalsize{Return this exam!  No calculators!}\\
\normalsize{Exam 1}\\
\makebox[4in][l]{Name:}
SSN:}
\date{}

\maketitle{}

\underline{For questions 1-4 assume a word size of 6 bits.}

\begin{enumerate}
\item {\bf (2 pt.)} Convert $110100_2$ to decimal.
\begin{description}
\item{a) }16
\item{b) }24
\item{c) }42
\item{d) }84
\item{e) }None of the above.
\end{description}

\item {\bf (2 pt.)} Convert $53_{10}$ to binary.
\begin{description}
\item{a) }$110101_2$
\item{b) }$010111_2$
\item{c) }$111111_2$
\item{d) }$100011_2$
\item{e) }None of the above.
\end{description}

\item {\bf (2 pt.)} Convert $39_{16}$ to decimal.
\begin{description}
\item{a) }$992_{10}$
\item{b) }$71_{10}$
\item{c) }$57_{10}$
\item{d) }$39_{10}$
\item{e) }None of the above.
\end{description}
\pagebreak
\item {\bf (1 pt.)} If overflow occurs indicate it, otherwise identify
the correct answer: $111010_2 + 011011_2$
\begin{description}
\item{a) }$100101_2$
\item{b) }$110101_2$
\item{c) }$111111_2$
\item{d) }$101011_2$
\item{e) }Overflow occurs.
\end{description}

\item {\bf (2 pt.)} How many different binary numbers can be formed
using $N$ bits?
\begin{description}
\item{a) } $\log_2(N)-1$
\item{b) } $N-1$
\item{c) } $N$
\item{d) } $2^{N}-1$
\item{e) } $2^N$
\end{description}

\item {\bf (1 pt.)}A logic function who's output equals 0 when any
input equals 1, describes which of the following?
\begin{description}
\item{a) } AND
\item{b) } OR
\item{c) } NAND
\item{d) } NOR
\item{e) } XOR
\end{description}

\item {\bf (3 pt.)} Does A(A+B) = A?
Hint, the answer is not c.
\begin{description}
\item{a) } Yes
\item{b) } No
\item{c) } Maybe?
\item{d) } None of the above.
\end{description}
\pagebreak
\underline{For questions 8-11 assume F(A,B,C,D)= A'C+B'(C'+D)'+AB'CD}

\item {\bf (2 pt.)} What does F(0,1,0,0) equal?
\begin{description}
\item{a) } 0
\item{b) } 1
\item{c) } B
\item{d) } B'
\item{e) } Not enough information.
\end{description}

\item {\bf (2 pt.)} What does F(1,B,1,0) equal?
\begin{description}
\item{a) } 0
\item{b) } 1
\item{c) } B
\item{d) } B'
\item{e) } Not enough information.
\end{description}

\item {\bf (2 pt.)} How many AND gates does it take to realize F 
as shown (do not simplify)?
\begin{description}
\item{a) } 1
\item{b) } 2
\item{c) } 3
\item{d) } 4
\item{e) } None of the above.
\end{description}

\item {\bf (2 pt.)} What is the \SOPmin expression for F?
\begin{description}
\item{a) } A'C + B'C' + A'D  + AB'CD
\item{b) } A'C + B'CD' + AB'CD
\item{c) } A'C + AB'C
\item{d) } A'C + B'C
\item{e) } C(A'+B')
\end{description}
\pagebreak
\underline{For questions 12,13 use the figure below.}
\begin{figure}[ht]
\rightline{\psfig{figure=./Fig1/cir_1.eps}}
\end{figure}

\item {\bf (2 pt.)} What is the symbolic representation of $F(A,B,C,D)$ 
\begin{description}
\item{a) } ((A'+B+C) +ABC)(ABC')'
\item{b) } ((A +B+C')'+A'BC)(A'BC)'
\item{c) } ((A'+B+C)'+ABC') (ABC')
\item{d) } ((A'+B+C)'+ABC')(ABC')'
\item{e) } None of the above.
\end{description}

\item {\bf (2 pt.)} What is F(1,1,0)=?
\begin{description}
\item{a) } 1
\item{b) } 0
\end{description}

\item {\bf (2 pt.)} You are working on a kmap and find a legal grouping 
of 16 1's.  How many variables does the function have?
\begin{description}
\item{a) } 4
\item{b) } 6
\item{c) } 8
\item{d) } 16
\item{e) } Not enough information.
\end{description}

\item {\bf (1 pt.)} A cell in a 6 variable kmap is adjacent to how many other cells?  
\begin{description}
\item{a) } 3
\item{b) } 6
\item{c) } 36
\item{d) } 64
\item{e) } None of the above.
\end{description}

\item {\bf (2 pt.)} How many different \SOPmin solutions exist for \\
F(A,B,C,D)=$\Sigma$m(0,2,5,6,7) ?
\marginpar{ \tiny $$ \begin{array} {c||c|c|c|c}
        A \bs BC & 00 & 01 & 11 & 10 \\ \hline \hline
        0        &    &    &    &    \\ \hline
        1        &    &    &    &    \\ 
\end{array} $$ }
\begin{description}
\item{a) } 1
\item{b) } 2
\item{c) } 3
\item{d) } 4
\item{e) } 5
\end{description}
\pagebreak
\item {\bf (2 pt.)} Determine the \SOPmin expression for \\
F(A,B,C,D)=$\Sigma$m(4,9,12,15)+$\Sigma$d(2,6,10,11,14)
\marginpar{ \tiny $$ \begin{array} {c||c|c|c|c}
        AB \bs CD & 00 & 01 & 11 & 10 \\ \hline \hline
        00        &    &    &    &    \\ \hline
        01        &    &    &    &    \\ \hline
        11        &    &    &    &    \\ \hline
        10        &    &    &    &    \\
\end{array} $$ }
\begin{description}
\item{a) } C'D' + AB'D+AC
\item{b) } BD' + AC + AB'D
\item{c) } BC'D' + AB'D + ABC
\item{d) } BC'D' + AB'D + AC
\item{e) } None of the above.
\end{description}

\item {\bf (3 pt.)} Determine the \SOPmin expression for \\
F(A,B,C,D)=(A+B+C'+D')(A+B'+D)(A'+B'+C)(A'+B+D')(A'+B'+D)
\marginpar{ \tiny $$ \begin{array} {c||c|c|c|c}
        AB \bs CD & 00 & 01 & 11 & 10 \\ \hline \hline
        00        &    &    &    &    \\ \hline
        01        &    &    &    &    \\ \hline
        11        &    &    &    &    \\ \hline
        10        &    &    &    &    \\
\end{array} $$ }
\begin{description}
\item{a) } B'D'+A'C'D+BCD
\item{b) } BD+ACD'+B'C'D'
\item{c) } BD+AC'D+B'CD
\item{d) } B'D'+A'CD'+BC'D'
\item{e) } None of the above.
\end{description}

\underline{For questions 19,20 assume that espresso has generated the following output.}
\begin{verbatim}
.i 3
.o 2
.ilb A B C
.ob F G
.p 3
1-0 10
01- 01
-01 11
.e
\end{verbatim}

\item{\bf (1 pt.)}  Which product term is shared.
\begin{description}
\item{a) } AC'
\item{b) } AB'
\item{c) } B'C
\item{d) } F and G
\end{description}

\item{\bf (1 pt.)}  Which of the following is equivalent to G(A,B,C)?
\begin{description}
\item{a) } G(A,B,C) = $\Sigma$m(1,2,3,5)
\item{b) } A'C+A'BC'+B'C
\item{c) } A'B + B'C
\item{d) } All of the above.
\end{description}

\item{\bf (5 pt.)} Explain how to convert a truth table into a
canonical SOP expression.  Your answer should address the meaning
and definition of minterms and their role in the final circuit.

\rule{112mm}{.1mm} \\
\rule{112mm}{.1mm} \\
\rule{112mm}{.1mm} \\
\rule{112mm}{.1mm} \\
\rule{112mm}{.1mm} \\
\rule{112mm}{.1mm} \\
\rule{112mm}{.1mm} \\
\rule{112mm}{.1mm} \\
\rule{112mm}{.1mm} \\
\rule{112mm}{.1mm} \\
\rule{112mm}{.1mm} \\
\rule{112mm}{.1mm} \\
\rule{112mm}{.1mm} \\
\rule{112mm}{.1mm} \\
\rule{112mm}{.1mm} \\
\rule{112mm}{.1mm} \\
\rule{112mm}{.1mm} \\
\rule{112mm}{.1mm} \\
\rule{112mm}{.1mm} \\
\rule{112mm}{.1mm} \\
\rule{112mm}{.1mm} \\
\rule{112mm}{.1mm} \\
\rule{112mm}{.1mm} \\
\rule{112mm}{.1mm} \\
\rule{112mm}{.1mm} \\
\rule{112mm}{.1mm} \\
\rule{112mm}{.1mm} \\
\rule{112mm}{.1mm} \\
\rule{112mm}{.1mm} \\
\rule{112mm}{.1mm} \\
\rule{112mm}{.1mm} \\
\rule{112mm}{.1mm} \\
\rule{112mm}{.1mm} \\
\rule{112mm}{.1mm} \\
\rule{112mm}{.1mm} \\
\rule{112mm}{.1mm} \\
\rule{112mm}{.1mm} \\
\rule{112mm}{.1mm} 
\end{enumerate}
\end{document}
