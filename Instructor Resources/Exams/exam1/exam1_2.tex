\documentclass{article}
\usepackage{epsfig, latexsym}

\begin{document}

\newcommand{\SOPmin}{${\rm SOP}_{\rm min} \ $}
\newcommand{\POSmin}{${\rm POS}_{\rm min} \ $}
\newcommand{\bs}{\backslash}


\title{
\Huge{CSE 271 -- Fall 2001}\\
\normalsize{Return this exam!  No calculators!}\\
\normalsize{Exam 1}\\
\makebox[4in][l]{Name:}
SSN:}
\date{}

\maketitle{}

\underline{For questions 1-5 assume a word size of 6 bits.}
\

\begin{enumerate}
\item {\bf (2 pt.)} Convert $101011_2$ to decimal.
\begin{description}
\item{a) }16
\item{b) }24
\item{c) }42
\item{d) }84
\item{e) }None of the above.
\end{description}

\item {\bf (2 pt.)} Convert $35_{10}$ to binary.
\begin{description}
\item{a) }$110101_2$
\item{b) }$010111_2$
\item{c) }$111111_2$
\item{d) }$100011_2$
\item{e) }None of the above.
\end{description}

\item {\bf (2 pt.)} Convert $35_{16}$ to binary.
\begin{description}
\item{a) }$011101_2$
\item{b) }$110101_2$
\item{c) }$100011_2$
\item{d) }$101011_2$
\item{e) }None of the above.
\end{description}
\pagebreak
\item {\bf (1 pt.)} If overflow occurs indicate it, otherwise identify
the correct answer: $111010_2 + 011011_2$
\begin{description}
\item{a) }$100101_2$
\item{b) }$110101_2$
\item{c) }$111111_2$
\item{d) }$101011_2$
\item{e) }Overflow occurs.
\end{description}

\item {\bf (1 pt.)} If overflow occurs indicate it, otherwise identify
the correct answer: $101011_2 + 010011_2$
\begin{description}
\item{a) }$011101_2$
\item{b) }$110101_2$
\item{c) }$111110_2$
\item{d) }$101011_2$
\item{e) }Overflow occurs.
\end{description}

\underline{For questions 6-8 assume F(A,B,C,D)= AB(C+B')+A'C+BD'}

\item {\bf (2 pt.)} What does F(0,1,0,0) equal?
\begin{description}
\item{a) } 0
\item{b) } 1
\item{c) } A
\item{d) } A'
\item{e) } Not enough information.
\end{description}

\item {\bf (2 pt.)} What does F(A,1,0,0) equal?
\begin{description}
\item{a) } 0
\item{b) } 1
\item{c) } A
\item{d) } A'
\item{e) } Not enough information.
\end{description}

\item {\bf (2 pt.)} How many AND gates does it take to realize F 
as is (do not simplify)?
\begin{description}
\item{a) } 3
\item{b) } 5
\item{c) } 7
\item{d) } 8
\item{e) } None of the above.
\end{description}
\pagebreak
\item {\bf (3 pt.)} What is the symbolic representation of $F(A,B,C,D)$ shown
as shown in the figure below?

\begin{figure}[ht]
\rightline{\psfig{figure=./Fig1/cir.eps,width=4in,clip=}}
\end{figure}
\begin{description}
\item{a) } ((A'+C')' + D + AB)'B
\item{b) } ((A+C)' + D + AB)'B
\item{c) } (A+C+B+AB)'B
\item{d) } (A+C'+D+AB)'B
\item{e) } (A'+C'+D+AB)'B'
\end{description}

\item {\bf (2 pt.)} If F(A,B,C,D)=AC'D'+BC'D', what does F' equal?
\begin{description}
\item{a) } AC'D'+BC'D'
\item{b) } (A'+C+D)(B'+C+D)
\item{c) } (AC'D')(BC'D')
\item{d) } (A'+C+D)(B'+C+D)
\item{d) } (A+C'+D')(B+C'+D')
\item{e) } None of the above.
\end{description}

\item {\bf (1 pt.)}Which of the following describes a normal
design process?
\begin{description}
\item{a) } WS $\rightarrow$ TT $\rightarrow$ CD $\rightarrow$ Sym
\item{b) } WS $\rightarrow$ Sym $\rightarrow$ TT $\rightarrow$ CD
\item{c) } WS $\rightarrow$ CD $\rightarrow$ Sym $\rightarrow$ TT
\item{d) } WS $\rightarrow$ TT $\rightarrow$ Sym $\rightarrow$ CD
\item{e) } WS $\rightarrow$ Kmap $\rightarrow$ CD
\end{description}

\item {\bf (2 pt.)} You are working on a kmap and find a legal grouping 
of 16 1's.  When you write the product term for this grouping it contains 3
variables.  How many variables does the function have?
\begin{description}
\item{a) } 4
\item{b) } 5
\item{c) } 6
\item{d) } 7
\item{e) } None of the above.
\end{description}
\pagebreak
\item {\bf (3 pt.)} Does AB'C + BC + A'BC' equal A'C + BC + A'B?
Hint, the answer is not c.
\begin{description}
\item{a) } Yes
\item{b) } No
\item{c) } Maybe?
\item{d) } None of the above.
\end{description}

\item {\bf (2 pt.)} How many rows does a truth table of 9 variables have?
\begin{description}
\item{a) } 81
\item{b) } 256
\item{c) } 512
\item{d) } 729
\item{e) } 1024
\end{description}

\item {\bf (2 pt.)} A cell in a 8 variable kmap is adjacent to how many 
other cells?  
\begin{description}
\item{a) } 3
\item{b) } 8
\item{c) } 16
\item{d) } 64
\item{e) } 256
\end{description}

\item {\bf (2 pt.)} Determine the \SOPmin expression for \\
F(A,B,C,D)=$\Sigma$m(4,7,9,10,12,13,14,15)
\marginpar{ \tiny $$ \begin{array} {c||c|c|c|c}
        AB \bs CD & 00 & 01 & 11 & 10 \\ \hline \hline
        00        &    &    &    &    \\ \hline
        01        &    &    &    &    \\ \hline
        11        &    &    &    &    \\ \hline
        10        &    &    &    &    \\
\end{array} $$ }
\begin{description}
\item{a) } BC'D'+AC'D+BCD+ACD'
\item{b) } AB+BC'D'+AC'D+BCD+ACD'
\item{c) } B'C'D'+A'C'D+B'CD+A'CD'
\item{d) } AB+A'BC'D'+AB'C'D+A'BCD+AB'CD'
\item{e) } None of the above.
\end{description}
\pagebreak
\item {\bf (3 pt.)}Determine the \SOPmin realization for F.

\marginpar{ \tiny $$ \begin{array} {c||c|c|c|c}
        A \bs BC & 00 & 01 & 11 & 10 \\ \hline \hline
        0        &    &    &    &    \\ \hline
        1        &    &    &    &    \\ 
\end{array} $$ }

\begin{tabular}{l|l|l||l}
A & B & C & F \\ \hline \hline
0 & 0 & X & X \\ \hline
0 & 1 & 1 & 0 \\ \hline
X & 1 & 0 & X \\ \hline
1 & 0 & 0 & 1 \\ \hline
1 & X & 1 & 1 \\ 
\end{tabular}
\begin{description} 
\item{a) } A'B'+BC'
\item{b) } A+B'+C'
\item{c) } C'+AC
\item{d) } AB'C' + AC
\item{e) } None of the above.
\end{description}

\item {\bf (4 pt.)} Determine the \SOPmin expression for \\
F(A,B,C,D)=(A+D')(A'+B'+C)(A'+C'+D')(A'+B+C+D)(A'+B+C'+D),
show your work.
\pagebreak
\item{\bf (8 pt.)} I would like you to discuss what the implications
\marginpar{ \tiny $$ \begin{array} {c||c|c|c|c}
        A \bs BC & 00 & 01 & 11 & 10 \\ \hline \hline
        0        &    &    &    &    \\ \hline
        1        &    &    &    &    \\
\end{array} $$ }
are for the "funny ordering" of the numbers at the top of the kmap.
Your answer should address the purpose of a kmap and the underlying
boolean algebra.  An example might be illustrative in your discussion.
Your essay should have a begining, middle and end, and use complete
sentenses written in the english language. \\

\rule{112mm}{.1mm} \\
\rule{112mm}{.1mm} \\
\rule{112mm}{.1mm} \\
\rule{112mm}{.1mm} \\
\rule{112mm}{.1mm} \\
\rule{112mm}{.1mm} \\
\rule{112mm}{.1mm} \\
\rule{112mm}{.1mm} \\
\rule{112mm}{.1mm} \\
\rule{112mm}{.1mm} \\
\rule{112mm}{.1mm} \\
\rule{112mm}{.1mm} \\
\rule{112mm}{.1mm} \\
\rule{112mm}{.1mm} \\
\rule{112mm}{.1mm} \\
\rule{112mm}{.1mm} \\
\rule{112mm}{.1mm} \\
\rule{112mm}{.1mm} \\
\rule{112mm}{.1mm} \\
\rule{112mm}{.1mm} \\
\rule{112mm}{.1mm} \\
\rule{112mm}{.1mm} \\
\rule{112mm}{.1mm} \\
\rule{112mm}{.1mm} \\
\rule{112mm}{.1mm} \\
\rule{112mm}{.1mm} \\
\rule{112mm}{.1mm} \\
\rule{112mm}{.1mm} \\
\rule{112mm}{.1mm} \\
\rule{112mm}{.1mm} \\
\rule{112mm}{.1mm} \\
\rule{112mm}{.1mm} \\
\rule{112mm}{.1mm} \\
\rule{112mm}{.1mm} \\
\rule{112mm}{.1mm} \\
\rule{112mm}{.1mm} \\
\rule{112mm}{.1mm} \\
\rule{112mm}{.1mm} 
\end{enumerate}
\end{document}
