\documentclass{article}
\usepackage{epsfig, latexsym}

\begin{document}

\newcommand{\SOPmin}{${\rm SOP}_{\rm min} \ $}
\newcommand{\POSmin}{${\rm POS}_{\rm min} \ $}
\newcommand{\bs}{\backslash}


\title{
\Huge{CMPEN 271 -- Fall 2012}\\
\normalsize{Return this exam!  No calculators!}\\
\normalsize{Exam 1}\\
\makebox[4in][l]{Name:} }
\date{}

\maketitle{}


\begin{enumerate}

\item {\bf (2 pts.)} Convert $010100_2$ to decimal.

\begin{tabular}{p{0.7in} p{0.7in} p{0.7in} p{0.7in} l}
a)20 & b)24  & c)40  & d)42  & e) none of the above
\end{tabular}

\item {\bf (2 pts.)} Convert $24_{10}$ to binary.

\begin{tabular}{p{0.7in} p{0.7in} p{0.7in} p{0.7in} l}
a) $010010_2$ & b) $100010_2$ & c) $001100_2$ & d) $011000_2$ & e) none of the above
\end{tabular}

\item {\bf (2 pts.)} Convert $24_{16}$ to binary.

\begin{tabular}{p{0.7in} p{0.7in} p{0.7in} p{0.7in} l}
a) $110010_2$ & b) $110100_2$ & c) $001100_2$ & d) $011000_2$ & e) none of the above
\end{tabular}

\item {\bf (2 pts.)} What is the largest decimal number that you can make with 8 bits?

\begin{tabular}{p{0.7in} p{0.7in} p{0.7in} p{0.7in} l}
a) 256 & b) 255 & c) 32 & d) $2^{8}$ & e) none of the above
\end{tabular}

\item {\bf (1 pts.)} When represented as 4-bit binary numbers does 8 + 8 
generate overflow?

\begin{tabular}{p{0.7in} l}
a) yes & b) no  c) 8 cannot be represented in 2's complement
\end{tabular}

\item {\bf (1 pt.)} How many 1's does the output column in a truth table for a 
5-input OR gate have?

\begin{tabular}{p{0.7in} p{0.7in} p{0.7in} p{0.7in} l}
a) 0 & b) 1 & c) 5 & d) $2^{5}-1$ & e) $2^5$
\end{tabular}

\item {\bf (2 pt.)} Which expression is equivalent to (A'+B)'(B+AC)?
\begin{description}
\item{a) } 0
\item{b) } 1
\item{c) } AB'C
\item{d) } AB' + AB'C
\item{e) } None of the above
\end{description}

\pagebreak

\underline{For questions 8-10 let F(A,B,C)= AB'C + (A+B')C'}

\item {\bf (2 pts.)} What does F(1,1,0) equal?

\begin{tabular}{p{0.7in} p{0.7in} p{0.7in} p{0.7in} l}
a) 0 & b) 1 & c) C & d) C' & e) none of these
\end{tabular}

\item {\bf (2 pts.)} What does F(0,0,C) equal?

\begin{tabular}{p{0.7in} p{0.7in} p{0.7in} p{0.7in} l}
a) 0 & b) 1 & c) C & d) C' & e) none of these
\end{tabular}

\item {\bf (1 pt.)} How many OR gates does it take to realize F
as is (do not simplify)?

\begin{tabular}{p{0.7in} p{0.7in} p{0.7in} p{0.7in} l}
a) 1 & b) 2 & c) 3 & d) 4 & e) none of these
\end{tabular}

\underline{Utilize the following truth table for problems 11,12.}

\begin{tabular}{c|c|c||c|c}
A & B & C & F & G  \\ \hline \hline
0 & 0 & 0 & 1 & 1  \\ \hline
0 & 0 & 1 & 0 & 0  \\ \hline
0 & 1 & 0 & 0 & 0  \\ \hline
0 & 1 & 1 & 1 & 0  \\ \hline
1 & 0 & 0 & 1 & 1  \\ \hline
1 & 0 & 1 & 0 & 1  \\ \hline
1 & 1 & 0 & 1 & 0  \\ \hline
1 & 1 & 1 & 1 & 0  \\
\end{tabular} 

\item {\bf (1 pt.)} What function is described by $\prod M(0,4,5)$?

\begin{tabular}{p{0.7in} p{0.7in} p{0.7in} p{0.7in} l}
a) F & b) F' & c) G & d) G' & e) none of the above
\end{tabular}


\item {\bf (1 pt.)} How many product terms does the canonical SOP expression 
for G' have?

\begin{tabular}{p{0.7in} p{0.7in} p{0.7in} p{0.7in} l}
a) 1 & b) 2 & c) 3 & d) 4 & e) 5
\end{tabular}

\underline{Utilize the following word statement for problems 13,14.}

Design a 4-input digital system where the input $A=a_3a_2a_1a_0$ represent 
a 4-bit binary number.  The output is the input divided by 3.  Fractional
answers should be rounded up to the nearest integer.

\item {\bf (1 pt.)}How many rows will have the output $11_2$?

\begin{tabular}{p{0.7in} p{0.7in} p{0.7in} p{0.7in} l}
a) 1 & b) 2 & c) 3 & d) 4 & e) 5
\end{tabular}

\item {\bf (1 pt.)}How many bits of output are needed?

\begin{tabular}{p{0.7in} p{0.7in} p{0.7in} p{0.7in} l}
a) 1 & b) 2 & c) 3 & d) 5 & e) None of the above.
\end{tabular}

\pagebreak

\underline{Utilize the following circuit diagram for problems 16,17.}

\begin{figure}[ht]
\rightline{\psfig{figure=./Fig1/cir_10.eps,width=3in,clip=}}
\end{figure}

\item {\bf (3 pts.)} What is the symbolic representation of 
$G(A,B,C)$ as shown?  \begin{description}
\item{a) } AB
\item{b) } ABC'
\item{c) } ABC'(A+AB)
\item{d) } (AB+C')(A+AB)
\item{e) } None of the above.
\end{description}

\item {\bf (1 pts.)} What does G(0,1,1) equal?

\begin{tabular}{p{0.7in} p{0.7in} l}
a) 0 & b) 1 & c) None of the above
\end{tabular}

\item {\bf (1 pt.)} How many distinct \SOPmin solutions  exist for \\
F(A,B,C)=$\Sigma$m(1,2,3,4,5)
\marginpar{ \small $$ \begin{array} {c||c|c|c|c}
        A \bs BC & 00 & 01 & 11 & 10 \\ \hline \hline
        1        &    &    &    &    \\ \hline
        0        &    &    &    &    \\
\end{array} $$ }

\begin{tabular}{p{0.7in} p{0.7in} p{0.7in} p{0.7in} l}
a) 1 & b) 2 & c) 3 & d) 4 & e) 5
\end{tabular}


\item {\bf (3 pt.)} Determine the \SOPmin expression for \\
F(A,B,C,D)=$\Sigma$m(1,5,6,7,11,12,14,15)
\marginpar{ \small $$ \begin{array} {c||c|c|c|c}
        AB \bs CD & 00 & 01 & 11 & 10 \\ \hline \hline
        00        &    &    &    &    \\ \hline
        01        &    &    &    &    \\ \hline
        11        &    &    &    &    \\ \hline
        10        &    &    &    &    \\
\end{array} $$ }

\begin{description}
\item{a) } ABC'D' + A'C'D + ACD + BC 
\item{b) } ABD' + A'C'D + ACD + BC 
\item{c) } ABC'D' + A'C'D + ACD + BC
\item{d) } ABD' + A'C'D + BC  
\item{e) } None of the above.
\end{description}

\item {\bf (3 pt.)} Determine the \SOPmin expression for \\
F(A,B,C,D)=$\Sigma$m(1,4,5,9,11,14) + $\Sigma$d(6,7,10,12)
\marginpar{\small $$ \begin{array} {c||c|c|c|c}
        AB \bs CD & 00 & 01 & 11 & 10 \\ \hline \hline
        00        &    &    &    &    \\ \hline
        01        &    &    &    &    \\ \hline
        11        &    &    &    &    \\ \hline
        10        &    &    &    &    \\
\end{array} $$ }


\begin{description}
\item{a) } AB'D + A'C'D + BD'
\item{b) } AB'D + BC'D' + A'C'D + A'B + BCD'
\item{c) } AB'D + ACD' + A'B + B'C'D
\item{d) } AB'D + BC'D' + A'C'D + A'B + BD'
\item{e) } None of the above.
\end{description}

\item {\bf (3 pt.)} Determine the \POSmin expression for \\
F(A,B,C,D)= B'D' + A'D' + BC'D + ACD

\begin{description}
\item{a) } (A+C'+D')(A'+B'+D)(B+C+D')
\item{b) } (A+C'+D')(A'+B'+D)(B+D')
\item{c) } (C'+D')(B+D')
\item{d) } (B+D)(A+D)(B'+C+D')(A'+C'+D')
\item{e) } None of the above.
\end{description}


\vspace{0.2in} 

\begin{tabular}{ll}

$ \begin{array} {c||c|c|c|c}
        AB \bs CD & 00 & 01 & 11 & 10 \\ \hline \hline
        00        &    &    &    &    \\ \hline
        01        &    &    &    &    \\ \hline
        11        &    &    &    &    \\ \hline
        10        &    &    &    &    \\
\end{array} $

& 

$ \begin{array} {c||c|c|c|c}
        AB \bs CD & 00 & 01 & 11 & 10 \\ \hline \hline
        00        &    &    &    &    \\ \hline
        01        &    &    &    &    \\ \hline
        11        &    &    &    &    \\ \hline
        10        &    &    &    &    \\
\end{array} $

\\
\end{tabular}

\vspace{0.2in} 

\item {\bf (2 pt.)} You are working on a kmap and find a legal grouping
of 8 1's which requires 3 variables to represent.  How many variables
does the function have?

\begin{tabular}{p{0.7in} p{0.7in} p{0.7in} p{0.7in} l}
a) 3 & b) 4 & c) 5 & d) 6 & e) Not enough information.
\end{tabular}


\item {\bf (1 pt.)} A cell in a 7 variable kmap is adjacent to how many other cells?

\begin{tabular}{p{0.7in} p{0.7in} p{0.7in} p{0.7in} l}
a) 4 & b) 6 & c) 8 & d) 10 & e) None of the above.
\end{tabular}


\end{enumerate}

\end{document}


