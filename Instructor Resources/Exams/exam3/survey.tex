%% This is an ABET survey
%% Edited 12/2005 to include program outcomes
%% Edited 12/2008 to remove program outcomes

%% \begin{center} 
%% CSE 271 -- Introduction to Digital Systems \\
%% Course Objectives Survey \\ 
%% Penn State Erie, The Behrend College
%% \end{center}

\small{
In order to assure the continued success of the Behrend ECE
program, I would appreciate your evaluation of whether or
not this course met its learning objectives.  All answers
will be kept anonymous and will be used for the future 
improvement of this course and the entire ECE program.
Please record your answers on the SCANTRON form.
Thanks for your help.
}

\begin{tabular}{p{0.25in}p{2.5in}|p{0.45in}|p{0.4in}|p{0.4in}|p{0.4in}|p{0.4in}|} \\ \cline{3-7}
    &  & 
{\scriptsize Strongly Disagree} A & 
{\scriptsize Disagree} B & 
{\scriptsize Neutral} C & 
{\scriptsize Agree} D & 
{\scriptsize Strongly Agree} E \\ \hline

\item & I understand how to convert numbers from one base to another and 
how to add and subtract numbers represented in binary and 2's complement 
form. 
 & & & & & \\ \hline

\item & I understand how to convert between a truth table, a circuit diagram,
boolean expression and a word statement.
 & & & & & \\ \hline

\item & I understand how to simplify logic expression into SOP or
POS minimal form with or without don't cares.
 & & & & & \\ \hline

\item & I understand how to use ESPRESSO to minimize combinational logic
functions.
 & & & & & \\ \hline

\item & I understand how adders, comparators, multiplexers and decoders 
are built and how they operate. 
 & & & & & \\ \hline

\item & I understand how D,T,SR,JK, latches, clock latches and flip flops 
are supposed to operate. 
 & & & & & \\ \hline

\item & I understand how registers, shift registers, counters, tri-state 
logic and RAMs are built and how they should operate. 
 & & & & & \\ \hline

\item & I understand how to design Finite State Machines using a dense 
or Ones Hot encoding. 
 & & & & & \\ \hline

\item & I understand how to implement complex digital systems using the 
datapath and control design approach. 
 & & & & & \\ \hline

\end{tabular}
