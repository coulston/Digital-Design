\documentclass{article}
\usepackage{epsfig, latexsym}

\begin{document}

\newcommand{\bs}{\backslash}


\title{
\Huge{CSE 271 -- Fall 2002}\\
\normalsize{Exam 2}\\
\makebox[4in][l]{Name:}
SSN:}
\date{}

\maketitle{}

\begin{enumerate}
\item {\bf (1 pt.)} How many inputs do the AND gates inside a 64:1 mux have?
\begin{description}
\item{a) }2
\item{b) }4
\item{c) }5
\item{d) }8
\item{e) }None of the above.
\end{description}

\item {\bf (1 pt.)}How many inputs do the AND gates inside a 4:16 decoder have?
\begin{description}
\item{a) }2
\item{b) }4
\item{c) }16
\item{d) }$2^{16}$
\item{e) }None of the above.
\end{description}

\item {\bf (1 pt.)}How many inputs does the OR gate inside a 4:16 decoder have?
\begin{description}
\item{a) }2
\item{b) }3
\item{c) }4
\item{d) }16
\item{e) }Trick question, a decoder does not have an OR gate.
\end{description}

\item {\bf (1 pt.)} How many 2:1 muxes are in an 8-bit register?
\begin{description}
\item{a) } 2
\item{b) } 3
\item{c) } 4
\item{d) } 8
\item{e) } 64
\end{description}

\item {\bf (1 pt.)} Assuming a word size of 5 bits, interpret 10010 as a 2's complement
number.
\begin{description}
\item{a) }34
\item{b) }-2 
\item{c) }-12 
\item{d) }2
\item{e) }None of the above.
\end{description}

\item {\bf (1 pt.)} Assuming a word size of 4 bits, determine the 2's complement
representation of -7.
\begin{description}
\item{a) }1101
\item{b) }1010
\item{c) }1011 
\item{d) }1001 
\item{e) }None of the above.
\end{description}

\item {\bf (1 pt.)} Which of $G,L,E$ (or a logical combination of them) 
must be connected to the sel input of the mux so that \\
\verb+if (X > Y) then Z = X else Z = Y;+

\psfig{figure=./Fig2/compare.eps}

\begin{description}
\item{a) } G
\item{b) } L
\item{c) } E
\item{d) } L or E
\item{e) } G or E
\end{description}

\item {\bf (1 pt.)} How many address lines does a 256kx32 RAM have?
\begin{description}
\item{a) } 9
\item{b) } 16
\item{c) } 18
\item{d) } 19
\item{e) } 32
\end{description}

\pagebreak
In the following question you are to complete the design of a counter with
the following truth table and circuit diagram.  All the remains is to route
the signals to the mux.  For each of the four questions use the following
letters to code your answer:
\begin{description}
        \item {a)} The output of the ADDER
        \item {b)} 0
        \item {c)} D
        \item {d)} The output of the REGISTER
        \item {e)} None of the above
\end{description}

\begin{tabular}{lr}

\begin{tabular}[!t]{l|l|l||l}
clk         & $C_1 C_0$ & D & $Q^+$ \\ \hline \hline
0,1,$\downarrow$ & xx   & x & Q     \\ \hline
$\uparrow$     & 00     & x & Q     \\  \hline
$\uparrow$     & 01     & x & 0     \\  \hline
$\uparrow$     & 10     & x & Q+1 mod 16  \\  \hline
$\uparrow$     & 11     & D & D     \\
\end{tabular}
&
\psfig{figure=./Fig2/counter.eps, height=2.5in.}
\end{tabular}

\item{\bf (1 pt.)} Which signal goes into the $y_0$ input of the mux?
\item{\bf (1 pt.)} Which signal goes into the $y_1$ input of the mux?
\item{\bf (1 pt.)} Which signal goes into the $y_2$ input of the mux?
\item{\bf (1 pt.)} Which signal goes into the $y_3$ input of the mux?

\pagebreak
\item {\bf (2 pt.)}A 4-bit (circular) shift register has the 
following truth table.  Determine the value on the $Q$ lines 
at time=85 assuming the sequence of inputs shown in the 
timing diagram. Note the D signal is the decimal representation
of the binary input.

\begin{tabular}{l|l|l||l}
clk		& $C_1 C_0$	& D & $Q^+$	\\ \hline
0,1,$\downarrow$& xx		& x & Q		\\ \hline
$\uparrow$ 	& 00		& x & Q		\\  \hline
$\uparrow$ 	& 01		& x & $Q_0$\&$Q>>1$	\\  \hline
$\uparrow$ 	& 10		& x & $Q<<1$\&$Q_3$	\\  \hline
$\uparrow$ 	& 11		& D & D		\\
\end{tabular}

\psfig{figure=./Fig2/shift-time.eps,height=1.5in.}

\begin{description}
\item{a) } 0001
\item{b) } 0010
\item{c) } 0101
\item{d) } 1010
\item{e) } none of the above
\end{description}

\item {\bf (2 pt.)}A 4-bit up/down counter has the 
following truth table.  Determine the value on the $Q$ lines 
at time=85 assuming the sequence of inputs shown in the 
timing diagram for question 13. 

\begin{tabular}{l|l|l||l}
clk		& $C_1 C_0$	& D & $Q^+$	\\ \hline
0,1,$\downarrow$& xx		& x & Q		\\ \hline
$\uparrow$ 	& 00		& x & Q		\\  \hline
$\uparrow$ 	& 01		& x & Q-1 mod 16\\  \hline
$\uparrow$ 	& 10		& x & Q+1 mod 16\\  \hline
$\uparrow$ 	& 11		& D & D		\\
\end{tabular}

\begin{description}
\item{a) } 0101
\item{b) } 0110
\item{c) } 0100
\item{d) } 0011
\item{e) } none of the above
\end{description}

Questions 15-17 deal with the state diagram below:

\psfig{figure=./Fig2/sd.eps, height=1.0in}

\item {\bf (1 pt.)} How many inputs and outputs does the FSM have?

\begin{description}
\item{a) }2 bits of input and 3 bits of output
\item{b) }2 bits of input and 2 bits of output
\item{c) }1 bit of input and 3 bits of output
\item{d) }1 bit of input and 2 bits of output
\item{e) }1 bit of input and 1 bit of output
\end{description}

\item {\bf (1 pt.)} Determine the entry in the state table marked with a *.
\marginpar{ \tiny $$ \begin{array} {c||c|c}
        cs \bs X & 0   &  1   \\ \hline \hline
        A        &     &     \\ \hline
        B        &     &     \\ \hline
        C        &     &  *  \\ \hline
        D        &     &     \\
\end{array} $$ }

\begin{description}
\item{a) }A,101
\item{b) }A,011
\item{c) }B,110
\item{d) }B,101
\item{e) }C,101
\end{description}

\item {\bf (1 pt.)} Which state(s), once entered, cannot be left (without
the aide of a reset).
\begin{description}
\item{a) }A and C
\item{b) }B
\item{c) }D
\item{d) }No such state(s) exists.
\item{e) }None of the above.
\end{description}

\pagebreak
\item {\bf (2 pt.)} Given the following state table and the state assignment,
determine the entry in the transition kmap marked with a *.
{\small
$$\begin{array}{lll}
$$\begin{array}{c||c|c}
        cs \bs X & 0   &  1  \\ \hline \hline
        A        & B,0 & A,0 \\ \hline
        B        & D,0 & A,0 \\ \hline
        C        & C,1 & B,1 \\ \hline
        D        & A,1 & C,1 \\ 
\end{array}$$
&
$$\begin{array}{c||c|c}
        state & Q_1 & Q_0    \\ \hline \hline
        A     & 1 & 0  \\ \hline
        B     & 0 & 0 \\ \hline
        C     & 1 & 1 \\ \hline
        D     & 0 & 1 \\
\end{array}$$
&
$$\begin{array}{c||c|c}
        Q_1 Q_0 \bs X & 0   &  1   \\ \hline \hline
        00       &     &     \\ \hline
        01       &     &     \\ \hline
        10       & *   &     \\ \hline
        11       &     &     \\
\end{array}$$\\
\end{array}$$}

\begin{description}
\item{a) }00,0
\item{b) }01,1
\item{c) }10,0
\item{d) }11,0
\item{e) }None of the above.
\end{description}

Given the transition kmap to the right, determine the MIEs
and OEs.
{\small
$$\begin{array}{cll}
$$\begin{array}{c||c|c|c|c}
        Q \bs X_1 X_0 & 00  & 01  & 11  & 10  \\ \hline \hline
        0             & 1,1 & 1,1 & 0,1 & 0,1 \\ \hline
        1             & 1,0 & 0,0 & 0,0 & 1,0 \\
\end{array}$$
&
$$\begin{array} {c||c|c|c|c}
        Q \bs X_1 X_0 & 00 & 01 & 11 & 10 \\ \hline \hline
        0             &    &    &    &    \\ \hline
        1             &    &    &    &    \\
\end{array}$$
&
$$\begin{array} {c||c|c|c|c}
        Q \bs X_1 X_0 & 00 & 01 & 11 & 10 \\ \hline \hline
        0             &    &    &    &    \\ \hline
        1             &    &    &    &    \\
\end{array}$$ \\
Q^+,Z & D= & Z= \\
\end{array}$$}
\item {\bf (2 pt.)} What does $D=$ ?
\begin{description}
\item{a) }$X_1'X_0' + Q'X_1' + QX_1X_0'$
\item{b) }$X_1'X_0' + Q'X_1' + QX_0'$
\item{c) }$Q'X_1' + QX_1X_0'$
\item{d) }$Q'X_1' + QX_0'$
\item{e) }$Q'$
\end{description}

\item {\bf (2 pt.)} What does $Z=$ ?
\begin{description}
\item{a) }$X_1'X_0' + Q'X_1' + QX_1X_0'$
\item{b) }$X_1'X_0' + Q'X_1' + QX_0'$
\item{c) }$Q'X_1' + QX_1X_0'$
\item{d) }$Q'X_1' + QX_0'$
\item{e) }$Q'$
\end{description}

\pagebreak
\item {\bf (5 pt.)} Show how to count the number of times the 8-bit
value KEY occurs in a 128x8 RAM.  Use the following code.
\begin{verbatim}
    for(i=0; i<128; i++)
        if (M[i] == KEY) then count++;
\end{verbatim}
You may assume that any counter is initialized to 0 and
operate acording to the truth table shown in problem 14. 
The\verb+KEY+ register is initialized to the proper value.
The RAM already contains data; you can hardwire the RAM 
control inputs.  You can use any other building blocks you 
need.  Your solution will be scored as follows:
\begin{description}
\item 3 points for correct data flow
\item 1 point for correct control manipulation
\item 1 point for correct wire sizes
\end{description}




\end{enumerate}
\end{document}
