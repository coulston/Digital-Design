\documentclass{article}
\usepackage{epsfig, latexsym}

\begin{document}

\newcommand{\SOPmin}{${\rm SOP}_{\rm min} \ $}
\newcommand{\POSmin}{${\rm POS}_{\rm min} \ $}
\newcommand{\bs}{\backslash}


\title{
\Huge{CSE 271 -- Fall 2001}\\
\normalsize{Exam 2}\\
\makebox[4in][l]{Name:}
SSN:}
\date{}

\maketitle{}

\begin{tabular}{llll}
\begin{tabular}{c||c}
D & Q+   \\ \hline
0 & 0 \\ \hline
1 & 1 \\
\end{tabular}
&
\begin{tabular}{c||c}
T & Q+   \\ \hline
0 & Q \\ \hline
1 & Q' \\
\end{tabular}
&
\begin{tabular}{c|c||c}
S & R & Q+   \\ \hline
0 & 0 & Q \\ \hline
0 & 1 & 0 \\ \hline
1 & 0 & 1 \\ \hline
1 & 1 & x \\
\end{tabular}
&
\begin{tabular}{c|c||c}
J & K & Q+   \\ \hline
0 & 0 & Q \\ \hline
0 & 1 & 0 \\ \hline
1 & 0 & 1 \\ \hline
1 & 1 & Q' \\
\end{tabular}
\\
\end{tabular}


\begin{enumerate}
\item {\bf (1 pt.)} How many 3:8 decoders are required to construct a 6:64 decoder?
\begin{description}
\item{a) }2
\item{b) }4
\item{c) }5
\item{d) }8
\item{e) }9
\end{description}

\item {\bf (1 pt.)}How many inputs do the AND gates inside a 2:4 decoder have?
\begin{description}
\item{a) }2
\item{b) }3
\item{c) }4
\item{d) }5
\item{e) }None of the above.
\end{description}

\item {\bf (1 pt.)}How many inputs does the OR gates inside a 16:1 mux have?
\begin{description}
\item{a) }2
\item{b) }3
\item{c) }4
\item{d) }16
\item{e) }Trick question, a mux does not have an OR gate.
\end{description}

\pagebreak{}
\item {\bf (1 pt.)} If a 4:16 mux is constructed from 1:2 decoders, how
many of the 1:2 decoders get the select line $S_3$?
\begin{description}
\item{a) } 1
\item{b) } 2
\item{c) } 4
\item{d) } 8
\item{e) } None of the above.
\end{description}

\item {\bf (1 pt.)} Assuming a word size of 5 bits, interpret 11010 as a 2's complement
number.
\begin{description}
\item{a) }-10
\item{b) }-6 
\item{c) }-5 
\item{d) }-4 
\item{e) }None of the above.
\end{description}


\item {\bf (1 pt.)} Assuming a word size of 4 bits, determine the 2's complement
representation of -5.
\begin{description}
\item{a) }1101
\item{b) }1010
\item{c) }1011 
\item{d) }1001 
\item{e) }Cannot be done in 5 bits.
\end{description}

\item {\bf (1 pt.)} Assuming a word size of 4 bits and a 2's complement representation,
compute  1101 - 0011
\begin{description}
\item{a) } 0110
\item{b) } 1010
\item{c) } 1000
\item{d) } Invalid answer because overflow occurs.
\item{e) } None of the above.
\end{description}

\pagebreak
\item {\bf (2 pt.)} Which of $G,L,E$ below must be connected to the 
sel input of the mux so that \\
\verb+if (X > Y) then Z = X else Z = Y;+

\psfig{figure=./Fig2/compare.eps,height=1.5in.}

\begin{description}
\item{a) } G
\item{b) } L
\item{c) } E
\end{description}

\item {\bf (1 pt.)} How many address lines does a 1024x32 RAM have?
\begin{description}
\item{a) } 5
\item{b) } 8
\item{c) } 10
\item{d) } 16
\item{e) } 32
\end{description}

\pagebreak
In the following question you are to complete the design of a counter with
the following truth atble and circuit diagram.  All the remains is to route
the signals to the mux.  For each of the four questions use the following 
letters to code your answer:
\begin{description}
	\item {a)} The output of the ADDER
	\item {b)} 0
	\item {c)} D
	\item {d)} The output of the REGISTER
	\item {d)} None of the above
\end{description}

\begin{tabular}{lr}

\begin{tabular}[!t]{l|l|l||l}
clk         & $C_1 C_0$ & D & $Q^+$ \\ \hline \hline
0,1,$\downarrow$ & xx   & x & Q     \\ \hline
$\uparrow$     & 00     & x & Q     \\  \hline
$\uparrow$     & 01     & x & 0     \\  \hline
$\uparrow$     & 10     & x & Q+1 mod 16  \\  \hline
$\uparrow$     & 11     & D & D     \\
\end{tabular}
&
\psfig{figure=./Fig2/counter.eps, height=2.5in.} 
\end{tabular}

\item{\bf (1 pt.)} Which signal goes into the $y_0$ input of the mux?
\item{\bf (1 pt.)} Which signal goes into the $y_1$ input of the mux?
\item{\bf (1 pt.)} Which signal goes into the $y_2$ input of the mux?
\item{\bf (1 pt.)} Which signal goes into the $y_3$ input of the mux?

\pagebreak

\begin{figure}[ht]
\centerline{\psfig{figure=./Fig2/ExTim.eps,width=5in.,clip=}}
\end{figure}

\item {\bf (1 pts.)} What is the value of $Q_1$ at time 72?
\begin{description}
\item{a) }0
\item{b) }1
\item{c) }Changing rapidly.
\item{d) }Input changed too close to the edge to tell.
\item{d) }None of the above.
\end{description}

\item {\bf (1 pts.)} What is the value of $Q_2$ at time 37?
\begin{description}
\item{a) }0
\item{b) }1
\item{c) }Changing rapidly.
\item{d) }Input changed too close to the edge to tell.
\item{d) }None of the above.
\end{description}

\item {\bf (4 pts.)} Show the interconnections between a mux, a 
register and a counter to implement the following circuit.
\verb+if (X > Y) then Z = X else Z = Y;+


\pagebreak
\item {\bf (4 pt.)}Describe the sequence of bits seen on the
$Q$ output.  Write down the output sequence until it starts to 
repeat.  Assume that $Q=0100$ initially.

\begin{tabular}{l|l|l|l||l|r}
clk         & $C_1 C_0$ & D & cin & $Q^+$ & comment \\ \hline \hline
0,1,$\downarrow$ & xx   & x & x   & Q     & hold     \\ \hline
$\uparrow$     & 00     & x & x   & Q     & hold     \\  \hline
$\uparrow$     & 01     & x & cin & $cin,Q_3,Q_2,Q_1$  & shift right \\  \hline
$\uparrow$     & 10     & x & cin & $Q_2,Q_1,Q_0,cin$  & shift left \\  \hline
$\uparrow$     & 11     & D & x   & D     & parallel load  \\ 
\end{tabular}

\psfig{figure=./Fig2/shift.eps,height=1.5in.} 

\end{enumerate}
\end{document}
