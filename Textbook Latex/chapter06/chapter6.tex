\chapter{Sequential Building Blocks}
\addcontentsline{lobb}{chapter}{\thechapter \hspace{0.2cm} Sequential Logic Building Blocks}
\label{chapter:Sequential Building Blocks}
\graphicspath{ {./chapter06/Fig} }

In order to design complex systems, a suitable
abstraction is required for building them.  This requirement stems from the
limitations of the human mind to only manage about a dozen items
at once.  A complex system containing hundreds of components
simply cannot be organized in one pass.  Instead, the components
are organized into larger units which compose the system. Thus,
the number of components need to be considered at a development stage is reduced
by an order of complexity. These intermediate units should have
a high utility and be modular.  In other words, they should be
useful and applicable in a wide variety of design situations.

As an example, consider the design problem of writing a technical
document describing the operation of a pacemaker for a human heart.
This process does not begin by thinking about the spelling of
the individual words, instead it makes more sense to first
draft an outline.  This outline is an abstraction of the
written document, it is a simplified representation of the final
written document.  In somewhat the same way as an author ignores
spelling issues when constructing the outline, a designer is not concerned with
the operation of AND and OR gates
when designing a digital-signal processing chip.

Clearly, choosing components, or as they are called ``basic
building blocks," which are reusable and have non-overlapping
functionality, result in a small number of highly useful
components.  The set of available building blocks has largely
been determined by the electronics industry which provides basic
blocks as off-the-shelf prepackaged components.  These
time-tested components have established themselves over the years
as the accepted language of hardware design.
Several sequential logic building blocks are examined, next.
The word ``sequential" in their name implies that these building
blocks are different from those presented in Chapter 4 because
they have memory.

Like the combinational building blocks shown in
Figure~\ref{4-fig:comboBBAsys} each of the sequential basic building
blocks have control and data inputs, and status and
data outputs.  In addition, being sequential devices, most
also have an edge-sensitive clock input.  First to be examined is
the most basic sequential basic building block, the register.
\needspace{14\baselineskip} % The page above is too full to
\section{The Register}
\begin{buildingblock}{Register}
    \index{register|(}
        \label{page:reg}
        \begin{tabular}{|l|p{3.5in}|} \hline
            Nomenclature:  & N-bit register                           \\ \hline
            Data Input:    & N-bits vector $D=d_{N-1} \ldots d_1 d_0$.          \\ \hline
            Data Output:   & N-bit vector $Q=q_{N-1} \ldots q_1 q_0$    \\ \hline
            Control:       & 1-bit $C$              \\ \hline
            Status:        & none                                   \\ \hline
            Others:        & 1-bit edge-sensitive clock.  1-bit asynchronous
            active low reset.            \\ \hline
            Behavior:      &
            \begin{tabular}{c|c|c|c||c||c}
                $reset$ & $clk$          & $C$ & $D$ & $Q^+$ & comment \\ \hline
                0     & x            & x & x & $0$   & reset   \\ \hline
                1     & 0,1,falling  & x & x & $Q$   & hold  \\ \hline
                1     & rising       & 0 & x & $Q$   &  hold \\ \hline
                1     & rising       & 1 & D & D     &  load \\
            \end{tabular} \\ \hline
        \end{tabular}
    \end{buildingblock}

    An N-bit register is very much like a wide D flip flop.  It samples
    its N data inputs, denoted $D$ on the rising edge of the clock input.
    Depending on the control input, $C$, the register either holds its current
    value when $C=0$ or loads the new value when $C=1$.  The stored value
    of the register is asserted on its output, denoted $Q$. The columns
    in the register's state table are organized from left to right, from
    highest priority to lowest priority.   Holding the
    asynchronous active low reset line to 0 causes the stored value
    and the outputs to remain at 0 regardless of the value on any other
    input; the reset input has priority over all other inputs.

    A timing diagram for a 4-bit register is shown in Figure~\ref{fig:sequentialBBRegTime}.
    The initial value of the register is arbitrarily set to $A_{16}=0001_2$.
    Since the value of $Q$ is represented using four bits, its value on the timing
    diagram is shown as a wide trace.  This reflects the fact that $Q$ is
    composed of many bits.  At time=10, a positive edge of the clock arrives
    with $C=1$, hence the register loads $D=5_{16}=0101_2$, as its new value.
    The fact that the $Q$ outputs changes slightly after time=10 is an
    acknowledgment that the circuit elements inside the register have
    propagation delay.  The goofy behavior of the $C$ input around time=20 has
    no effect on the $Q$ outputs of the register because the clock is not rising.
    The rising clock edge at time=30 does not change the stored value of the
    register because $C=0$, hence the register holds its stored value.  The change
    in the $Q$ output at $time 50$ results from the rising clock edge and $C=1$.

    \begin{figure}[ht]
        \center{\includegraphics{RegTime}}
        \caption{A timing diagram for a 4-bit register.}
        \label{fig:sequentialBBRegTime}
    \end{figure}
    \index{timing!register}
\index{register|)}

An N-bit register is constructed using N, D flip flops.  A common error
committed by beginning students, and even some text books, is to AND the
$clk$ and $C$ signals together, sending the AND gate output to the
clock input of a D flip
flop.  This technique is incorrect because it causes the D flip flops to
sample their input when the $clk = 1$ and $C$ rises, contrary to the behavior
described in the register's state table.  As a general rule,
avoid modifying the clock signal unless it is absolutely necessary.

The correct construction of an N-bit register is shown in
Figure~\ref{fig:sequentialBBreg}.   Two modes are present for this circuit,
corresponding to $C=0$ and $C=1$.  When $C=1$, the four multiplexers
shown in Figure~\ref{fig:sequentialBBreg}, all route the data input $D_i$
to the input of the D flip flop.  When a clock edge arrives,
each $D_i$ is loaded into its respective flip flop and
soon thereafter appears on the $Q$ output.

\begin{figure}[ht]
    \center{\includegraphics{reg}}
    \caption{The internal organization of a 4-bit register.}
    \label{fig:sequentialBBreg}
\end{figure}

When $C=0$, the four multiplexers shown in Figure~\ref{fig:sequentialBBreg},
all route their data output $Q_i$ back to the input of the
D flip flop.  When a clock edge arrives, each $Q_i$ is
loaded into its respective flip flop and soon thereafter appears
on the $Q$ output. Thus, the $Q$ outputs appear to have
held their output value even though the internal
D flip flops have loaded a value.

\index{register|)}

\section{The Shift Register}
\index{register!shift|(}
A shift register is a register with the additional capability
of shifting its stored bits to the left or to the right.  The input,
output, and behavior of a shift register are shown in the
following table.

\begin{buildingblock}{Shift Register}
    \begin{tabular}{|l|p{3.5in}|} \hline
        Nomenclature:  & N-bit shift register with parallel load     \\ \hline
        Data Input:    & N-bits vector $D=d_{N-1} \ldots d_1 d_0$.          \\ \hline
        Data Output:   & N-bit vector $Q=q_{N-1} \ldots q_1 q_0$    \\ \hline
        Control:       & 2-bits $C=c_1 c_0$              \\ \hline
        Status:        & none                                   \\ \hline
        Others:        & 1-bit edge-sensitive clock.  1-bit asynchronous
        active low reset.                       \\ \hline
        Behavior:      &
        \begin{tabular}{c|c|c|c||c||c}
            $reset$ & $clk$          & $C$  & $D$ & $Q^+$ & comment \\ \hline
            0     & x            & xx & x & $0$   & reset   \\ \hline
            1     & 0,1,falling  & xx & x & $Q$   & hold  \\ \hline
            1     & rising       & 00 & x & $Q$   &  hold \\ \hline
            1     & rising       & 01 & x & $Q>>1$   &  shift right \\ \hline
            1     & rising       & 10 & x & $Q<<1$   &  shift left \\ \hline
            1     & rising       & 11 & x & D     &  load \\
        \end{tabular} \\ \hline
    \end{tabular}
    \label{page:shi}
\end{buildingblock}

If $Q=0110$ then shifting $Q$ to the left, denoted $Q<<1$,
yields 1100.  The symbol ``$<<$" denotes a shift left and the ``1"
describes how many bits to shift.  Shifting the original
value of $Q$ to the right by one bit, denoted $Q>>1$, yields
0011.  The ``$>>$" symbol denotes a shift right and the ``1"
describes how many bits.

Shifting is used to examine bits one at a time and in the
multiplication and division of binary numbers.  The bits
could be examined by looking at the LSB of a shift register
as it shifted its bits successively to the right.
Multiplication and division involve a bit more explanation.

For instance, consider multiplying a 4-bit binary number
$X=x_3 x_2 x_1 x_0$  by 2.  This task is accomplished by shifting
$X$ to the left one bit, yielding $X<<1=x_3 x_2 x_1 x_00$.  That
is a ``0" is place in the LSB.
In order to verify this, write down the decimal equivalent of
the shifted value of $X<<1$, $x_3*2^4 +x_2*2^3 +x_1*2^2 +x_0*2^1$.
Now, factor a 2 from each component of the sum, yielding
$2*(x_3*2^3 +x_2*2^2 +x_1*2^1 +x_0*2^0)$.  But this is $2*X$.

For each shift left by one bit, each of the exponents in the
decimal representation of $X$ increases by 1, adding a factor of
2 to every term of $X$ which can be factored out.  Hence, every
shift left increases $X$ by a factor of 2.  These factors
accumulate for each shift, so that shifting $X$ left three bits
increases $X$ by a factor of $2^3=8$.  Shifting can be used to
multiply by constants which are not powers of two by rewriting the
constant as the sum of powers of two.  For example, \label{page:MulyBy10}
to multiple a binary number $X$ by 10, rewrite $10$
as $8+2$ yielding $10X = (8+2)X = 8X+2X$.  So $10*X$ is computed
by adding together X shifted left by three bits to X shift left by one bit.

Shifting left may create a result
which cannot fit in the prescribed word-size.  For example, if
$X=12_{10}=1100_2$ is shifted left one bit in a 4-bit shift register,
the result $1000_2 = 8_{10}$ does not represent $24_{10}$ because
this value cannot fit into four bits.  It is easy to see a shift left
results in overflow whenever the MSB equals 1.

Dividing binary numbers by powers of two is accomplished by
shifting the bits to the right.  Since it is possible that division by
two results in a fraction and combined with the fact that binary numbers
represent integers, some form of rounding must occur.  For example,
if $X=5_{10}=0101$ is shifted right by one bit, the result is
$010_2=2_{10}$.  The 1 in the LSB of $X$ is lost.  Hence, shifting
to the right divides a number by 2 and rounds down when there
is a fractional result.

In order to accommodate, the diverse set of situations
when shifting can be used, three types of shifts are available:
arithmetic, logical, and circular.  Since each of these can
occur to the left or right, then six possible effects are to be considered due to
shifting a 4-bit string $x_3 x_2 x_1 x_0$.  These are enumerated
in the following table.

\begin{tabular}{c|c|c}

    & Left            & Right            \\ \hline
    Arithmetic    & $x_2 x_1 x_0 0$        & $x_3 x_3 x_2 x_1$ \\ \hline
    Circular    & $x_2 x_1 x_0 x_3$    & $x_0 x_3 x_2 x_1$ \\ \hline
    Logical     & $x_2 x_1 x_0 0$        & $0 x_3 x_2 x_1$   \\

\end{tabular}

All these shifts are characterized by how they ``fill in" the void created
by the shift.  Logical shifts always fill in the void with a 0 and are
used mainly for multiplication and division of binary numbers.

Circular shifts fill in the void with the bit that ``falls off" from the other
end of the shift.  For example, circularly shifting 0101 to the right yields
1010 because the 1 that falls off the LSB is inserted into the void created
at the MSB.  Circular shifts are useful when each bit of a register needs to be
inspected without destroying the registers contents.

Arithmetic shifts are used mainly to manipulate 2's-complement
numbers.  An arithmetic shift right fills the void with a duplicate of
the MSB, maintaining the ``sign" of the 4-bit 2's-complement number.
This process is the same as the one governing the sign-extending of
2's-complement numbers as discussed on page~\pageref{page:2sPad}. An
arithmetic shift left fills in the void with a 0 because this multiplies
both positive and negative quantities by 2.

To better understand its organization
the design of a 4-bit circular shift register that holds
its value, circularly shifts its contents to the right, circularly
shifts its contents to the left, or loads an external 4-bit input
is examined.
Since this circular shift register has four functions, it requires
two bits of control, denoted $c_1 c_0$.  The assignment of bit values to
the various functions is arbitrary and defined in the table below.
The external 4-bit data input will be denoted $D$.  A clock signal,
$clk$, indicates when the circular shift register should perform
its function.  Finally, the 4-bit output from the circular shift
register is denoted $Q$.

\begin{tabular}{c|c|c||c||c}

    $clk$          & $C$  & $D$  & $Q^+$     & comments     \\ \hline
    0,1,falling  & x  & x  & $Q$       &        \\ \hline
    rising       & 00 & x  & $Q$       & hold    \\ \hline
    rising       & 01 & x  & $Q(0)|Q>>1$ & CSR    \\ \hline
    rising       & 10 & x  & $Q<<1|Q(3)$ & CSL    \\ \hline
    rising       & 11 & D  & D         & load    \\

\end{tabular}

The notation in $Q^+$ column of the state table needs some further
explanation.  The $Q(0)$ symbol refers to the LSB of $Q$, the $|$ symbol
denotes concatenation, the merging of the bits to the left and right
of the $|$  symbol.  Thus, the expression $Q(0) | Q>>1$ means that the
LSB of $Q$ should be ``glued" to the most significant three bits of
$Q$.

The circuit for the circular shift register is shown in
Figure~\ref{fig:sequentialBBShiftReg} and consists of two major components, D flip
flops and 4:1 muxes.

\begin{figure}[ht]
    \center{\includegraphics{ShiftReg}}
    \caption{The internal organization of a 4-bit circular shift register.}
    \label{fig:sequentialBBShiftReg}

\end{figure}

The organization of the shift register is similar to the register, the
difference being the larger mux.  When the 2-bit control signal,
denoted by the slash with a 2 through it, is 00, $Q_i$ is routed to
the input of each D flip flop.  The rising edge of the clock
causes each D flip flop to latch its previous output, causing no
change in the outputs.  When $C=01$, the data input to each D flip
flop is $Q$, circularly shifted to the right.  This movement is denoted by
writing the name of each $Q$ bit on the mux input instead of drawing
the lines because otherwise the diagram quickly becomes messy and confusing.
Hence, when on the rising edge of the clock, the D flip flops
latch the shifted value of the outputs, making all the outputs
appear to circularly shift to the right, one bit.  The $C=10$ input
cause the inputs to circularly shift to the left.  The $C=11$ input,
sometimes called a parallel load because all four bits are loaded
simultaneously, loads each bit of the external 4-bit input, $D$, to its
respective flip flop.
\index{register!shift|)}

\section{The Counter}
\index{counter|(}
A counter is a simple but surprisingly versatile piece of hardware.
Its behavior is obvious from its name; it counts up when instructed
to do so.

\begin{buildingblock}{Counter}
    \begin{tabular}{|l|p{3.5in}|} \hline

        Nomenclature:  & N-bit counter with parallel load                  \\ \hline
        Data Input:    & N-bits vector $D=d_{N-1} \ldots d_1 d_0$.          \\ \hline
        Data Output:   & N-bit vector $Q=q_{N-1} \ldots q_1 q_0$    \\ \hline
        Control:       & 2-bits $C=c_1 c_0$              \\ \hline
        Status:        & none                                   \\ \hline
        Others:        & 1-bit edge-sensitive clock.  1-bit asynchronous
        active low reset.                       \\ \hline
        Behavior:      &

        \begin{tabular}{c|c|c|c||c||c}

            $reset$ & $clk$          & $C$  & $D$   & $Q^+$  & comment     \\ \hline
            0     & x            & xx & x   & $0$    & reset       \\ \hline
            1     & 0,1,falling  & xx & x   & $Q$    & hold        \\ \hline
            1     & rising       & 00 & x   & $Q$    & hold        \\ \hline
            1     & rising       & 01 & x   & $D$    & load        \\ \hline
            1     & rising       & 10 & $D$   & $Q+1$  & count up    \\ \hline
            1     & rising       & 11 & x   & x      &             \\

        \end{tabular}    \\  \hline
    \end{tabular}
    \label{page:counter}
\end{buildingblock}

When the 2-bit control input equals 10 and a clock edge arrives, the
counter counts up.  Since number of bits are limited, the
counting will at some point overflow.  When this happens, the count
value rolls-over back to 0 and begins counting up again.
For example, a 4-bit counter rolls-over to 0 when it tries
to count up at 15.  This behavior is similar to the behavior of the digits of
a car's odometer rolling over from 9 to 0.

A timing diagram for a 4-bit counter is shown in Figure~\ref{fig:sequentialBBCountTime}.
The initial value of the counter was arbitrarily set to $E_{16}=1110_2$.
At time=10, a positive edge of the clock arrives.  Since the $C$ input
is equal to $10_2$ at time=10, then the counter counts up to $F_{16}=1111_2$.
The goofy behavior of the $C$ input between time=10 and time=20 has
no effect on the $Q$ outputs of the counter because the clock is not rising.
At time=30, the $C$ input is equal to $00_2$ so the counter holds the current
count value.  At time=50, the $C$ input is equal $10_2$ so the counter counts up
rolling over to $0_{16}$. At time=70 the counter counts up to $1_{16}=0001_2$.
At time=90, the counter loads $7_{16}=0111_2$.  Notice, as in
Figure~\ref{fig:sequentialBBRegTime}, the counter timing diagram shows a small
propagation delay for the $Q$ output.

\begin{figure}[ht]
    \center{\includegraphics{CountTime}}
    \caption{A timing diagram for a 4-bit counter.}
    \label{fig:sequentialBBCountTime}
\end{figure}
\index{timing!counter}

The internal organization of a counter is very similar to the structure of
the register and shift register.  A set of D flip flops holds the current
count value.  A mux decides which input is presented to the D flip flops
to be latched up when the positive clock edge arrives.  An adder is
used to add 1 to the outputs of the flip flops (current count value).
If the overflow output of the adder is ignored, then the adder has the
advantage of rolling over to 0 when the count value is at the
maximum.  It is easy to verify that 111...1 + 1 = 000...0.  The
entire circuit is shown in Figure~\ref{fig:sequentialBBcounter}.

\begin{figure}[ht]
    \center{\includegraphics{counter}}
    \caption{The internal organization of a 4-bit counter.}
    \label{fig:sequentialBBcounter}
\end{figure}

Notice, the data inputs to the mux shown in Figure~\ref{fig:sequentialBBcounter}
all have a slash through them with
the number 4.  This notation indicates each of the data inputs is a four
bit wide
vector, and consequently, this device is a multibit mux, as discussed on
page~\pageref{page:wmu}.  Additionally, the $y_3$ input is unused.  This
inefficiency is the cost of using a generalized building block.
\index{counter|)}

%\section{The Read Only Memory}
%\pagebreak

\section{The Static RAM}
Random Access Memory (RAM) is an unfortunate name for a digital circuit
that has nothing random about it.  The name ``RAM" originated in the early
days of computing when engineers used cassette tapes as an inexpensive way
to store ``lots" of data.  One of the downfalls of a cassette tape is the
sequential nature of data access.  That is, the time to access a piece of
data depends on its position on the tape and the current position of the
tape.  RAMs do not suffer from the limits of a sequential access devices; the
amount of time to access any \textit{ random location} is the same.

In order to understand how RAMs work, you will need to understand a few
concepts first, so let's start.

A RAM is a device that stores and retrieves bundles of bits, called
a word, from an address.  You can envision a RAM as tower
of binary numbers like that shown in Figure~\ref{fig:sequentialBBram}.
The width of a RAM is called the \textbf{word size}.  In Figure~\ref{fig:sequentialBBram}
the word size is 4-bits.  Each set of 4-bits is called a word.  In
Figure~\ref{fig:sequentialBBram}, the words are organized in rows.
Each word in the RAM has an address. By convention addressing
starts at location 0.  In Figure~\ref{fig:sequentialBBram}, the word at
address 5 is 0011.  Retrieving information from a RAM is called a
\textbf{read} operation.  Storing information into a RAM is called a \textbf{write}
operation.  The terms read/write should invoke the idea that the RAM is a
book or ledger that you are reading with your eyes or writing into it with a pen.

\begin{figure}[ht]
\center{\includegraphics{ram}}
\caption{A 8x4 RAM has eight words each containing four bits.  The addresses
are shown to the left.}
\label{fig:sequentialBBram}
\end{figure}

While there is no relationship between the word size and the number of
words stored in the RAM, there is a relationship between the number
of words stored in the RAM and the number of address bits:
The number of bits in the address must be sufficient to assign
each word a unique binary address.  In Figure~\ref{fig:sequentialBBram}, the
addresses range from 0 to 7, hence the address is a 3-bit value.   In general,
if a RAM has $2^N$ words, it requires a  $N$ bit address so that each word
is assigned a unique/distinct address.

The number of words in a RAM is often described using the metric system notation.

\begin{tabular}{ccc}
1k & $2^{10}$ & kilo \\
1M & $2^{20}$ & mega \\
1G & $2^{30}$ & giga \\
1T & $2^{40}$ & tera \\
\end{tabular}

The metric names are only close approximations to the actual, binary values
they describe.  For example, 1 kilometer is a 1,000 meters, but
1k bytes of memory is $2^{10}$ bytes, or $1024$ bytes.  The number of
address bits can be determined quickly from the metric abbreviations.
For example, a 256k RAM has $256k=2^8*2^{10}=2^{18}$ words, or 18 bits
of address.

A wide variety of random access memories are available to meet the wide
variety of applications in which users need to store data.  For situations where
data needs to be retained even though power is removed, non-volatile memories are employed.
These memories typically trade-off access and storage speed for the
convenience of non-volatility.  Volatile memories, memories which lose
their contents when power is removed, can be either static/dynamic, or
synchronous/asynchronous.

Dynamic memories store data on tiny capacitors and
consequently require periodic refreshing in order to retain their values.  These versions
are typically the highest density memories available, but the refresh circuitry
adds to their complexity.  Static memories store data in an arrangement
of transistors which does not need refreshing.  Static memories generally
faster, more expensive, and consume less power than their dynamic counterparts.

Access to a synchronous memory requires a clock and is reminiscent of a flip flop.
Asynchronous memories require the control and data signals be applied
for certain minimum duration in order for the operations to take effect.

In order to convey the behavior and utilization of a RAM we consider a popular
type of RAM used in many FPGAs, synchronous static RAM.
The input, output, and behavior of a synchronous static RAM is defined by the following table.

\index{RAM|(}
\begin{buildingblock}{RAM}
    \begin{tabular}{|p{2cm}|p{12cm}|} \hline

        Nomenclature:  & NxM RAM (random access memory)    \\ \hline
        Data Input:    &  M-bit vector $D=d_{M-1} \ldots d_1 d_0$
        $log_2(N)$-bit address $A=a_{log_2(N)-1} \ldots a_1 a_0$ \\ \hline
        Data Output:   & M-bit vector $D=d_{M-1} \ldots d_1 d_0$     \\ \hline
        Control:       & 1-bit $enb$ (read enable), 1-bit $wen$ (write enable),        \\ \hline
        Status:        & none                                   \\ \hline
        Others:        & 1-bit edge sensitive clock                 \\ \hline
        Behavior:      & \vspace{0.2cm}
        \begin{tabular}{c|c|c|c|c|c||l|l}
            clk            &    enb        &    wen        &    $A$    &    $D_{in}$    &   $D_{out}^+$
            &    Internal  & Note                          \\ \hline
            0, 1 falling        &     x         &    x         &     x     &     x         &    $D_{out}$
            &             &                  \\ \hline
            rising         &     0        &    0         &     x    &     x         &    $D_{out}$        &
            & hold                     \\ \hline
            rising         &     1        &    0         &     $A$    &     x         &    ram[$A$]        &
            & read                     \\ \hline
            rising         &     0        &    1         &     $A$    &     $D_{in}$     &    $D_{out}$
            & ram[$A$] $\Leftarrow D_{in}$      & write \\ \hline
            rising         &     1        &    1         &     $A$    &     $D_{in}$     &    $D_{in}$
            & ram[$A$] $\Leftarrow D_{in}$     & read/write \\
        \end{tabular}
        \vspace{0.2cm} \\ \hline

    \end{tabular}
    \label{page:ram}
\end{buildingblock}

there are many subtleties burred in this truth table that we will explore in the following example.  Before
we delve into this example,
take a close look at the column headed with $D_{out}^+$  The ``+'' reference in this column is meant to
elicit ideas we explored
in Chapter 5, the next state of a basic memory element.  In this case, the ``+'' in $D_{out}^+$ means that
this is the value of the
$D_{out}$ signal after the clock edge.  A digital designer can implement this behavior using an output register to
hold the value of Dout.  This output register is enabled anytime there is a read operation, enb = 1.

A write operation changes the internal values stored inside the RAM which is not always explicitly obvious.  Hence a
``Internal'' column was included to make these changes clear.  During write operations, the the RAM location given by
the addr input has its contents modified to the value on Din.

Let's examine how the truth table for a RAM is applied to the timing diagram shown in
Figure~\ref{fig:sequentialBBRamReadWrite}.

\begin{landscape}
    \begin{figure}[ht]
        \center{\includegraphics[width=\textheight]{wavedromRamReadWrite}}

        \begin{tikzpicture}[overlay]
            \draw[red,ultra thick,rounded corners]             (-4.2,0.4) rectangle (-4,8.6);
            \draw[orange,ultra thick,rounded corners]         (-3.5,0.4) rectangle (-3.3,8.6);
            \draw[lime,ultra thick,rounded corners]         (-2.1,0.4) rectangle (0.8,8.6);
            \draw[green,ultra thick,rounded corners]         (2.0,0.4) rectangle (2.2,8.6);
            \draw[blue,ultra thick,rounded corners]         (3.3,0.4) rectangle (6.3,8.6);
        \end{tikzpicture}

        \caption{The behavior of a 8x4 RAM whose initial values are given in Figure~\ref{fig:sequentialBBram}.
        Time is in units of nanoseconds (ns).}
        \label{fig:sequentialBBRamReadWrite}
    \end{figure}

    \color{red}
    \textbf{time = 3ns} Just before the rising edge of the clock  enb=1, wen=0 so this is a read operation.
    Since addr = $101_2 = 5_{10}$ on the rising edge of the clock, the value contained in ram[5] (which is
    0001) is latched into dout after the rising edge of the clock.

    \color{orange}
    \textbf{time = 4ns} Just before the rising edge of the clock  enb=0, wen=1 so this is a write operation.
    Since addr = $011_2 = 3_{10}$ on the rising edge of the clock, the value of din (which is 1110) is stored
    in ram[3]  after the rising edge of the clock.

    \color{lime}
    \textbf{time = 6ns to ...10ns}  During this interval,  enb=1, wen=0 on consecutive clock edges so this
    is a series of read operations.  The address is incremented from 2 to 5, so dout get the contents of
    ram[2] through ram[5].

    \color{green}
    \textbf{time = 12ns} Just before the rising edge of the clock  enb=1, wen=1 so this is a simultaneous
    read/write operation.
    Since addr = $110_2 = 6_{10}$ on the rising edge of the clock, the value of din (which is 1010) is stored
    in ram[6] and latched into dout after the rising edge of the clock.

    \color{blue}
    \textbf{time = 14ns...18ns} During this interval, enb=0, wen=1 on consecutive clock edges so this
    is a series of write operations.  The address is incremented from 0 to 3, so ram[0] through ram[2] have
    their values changed on consecutive clocked edges.

    \color{black}

\end{landscape}

Cases occur when it is necessary to build a larger RAM from several
smaller RAMs.  RAMs have two dimensions which can be increased: (1) the word size or
(2) the number of words.

\subsection{Increase the word size of a RAM}
Increasing the word size of a RAM is fairly straightforward because
each RAM chip gets all the address lines and handles some portion of
the data lines.  Structurally, this transformation is accomplished by
placing several RAM chips side-by-side. For example, in order
to construct a 256kx32 RAM from 256kx8 RAM chips, then four, 256kx8 RAM are
placed side-by-side as shown in Figure~\ref{fig:sequentialBBram}.

\begin{figure}[ht]
    \center{\includegraphics{wide}}
    \caption{The construction of a 256kx32 RAM from 256kx8 RAM chips.}
    \label{fig:sequentialBBwide}
\end{figure}

Each of the four chips have the same address lines and control
lines $enb$, and $wen$.  Thus, all the RAM chips behave in exactly the same
fashion, each handling its own set of eight bits.  Their actions are coordinated
by the common address and control signals.

\subsection{Increase number of words stored in RAM}
Combining RAM chips to increase the number of words involves some manipulation
of addresses.  For example, consider the problem of using 64kx8 RAM chips to
construct a circuit which behaves like a 256kx8 RAM.  Stacking four, 64kx8 RAM
chips above one another would create the required depth of 256k words.  However,
each of the 64k RAMs would have 16 address lines, but the 256k RAM being
constructed requires 18 address lines.  How is this discrepancy of the extra
two address bits resolved?  The solution depends on whether you are
performing a read or write operation.

Figure~\ref{fig:sequentialBBdeep} shows how a write operation is performed.
The 18 bits of address are split into two
components; the lower 16 bits are sent to all four of the 64kx8 RAMs
and the upper two bits are used to decide which of the four 64k RAM chips
is being written.  A decoder uses the upper two address bits as select, and routes
the $wen$ signal to one of the four 64k RAMs.

The read operation is not shown in Figure~\ref{fig:sequentialBBdeep} and is left
as an exercise.  The solution partitions the address in the same way as the
write operation.  But instead of a decoder, a multiplexer is used.  In order to
latch dout, a register also needs to be added to the output.

\begin{figure}[ht]

    \center{\includegraphics{deep}}
    \caption{An, incomplete 256kx4 RAM from 64kx4 RAM chips.}
    \label{fig:sequentialBBdeep}

\end{figure}
\index{RAM|)}

\section{Register Transfer}
A digital system designed using the datapath and control approach
transforms data into some predetermined sequence.  For now, focus
on the task of transforming the data performed by the datapath.
The datapath is composed of the basic building blocks discussed in
Chapters 4 and 6; their input, output and behavior is summarized on
page~\pageref{page:boxlist}.  Although a gross simplification, a
datapath can be considered as some combinational logic ``sandwiched"
between registers.  The clock signal to the datapath governs how
data moves in the datapath. After the rising edge of the clock, the
register outputs become valid. The output data from the registers
flows into the combinational logic which transforms this input into
an output.  The outputs of the combinational logic flow into the
register inputs.  The next rising edge of the clock causes the
registers to latch these new values.  This process then proceeds
into the next clock cycle.  In order to understand this discussion,
consider the simplified datapath shown in Figure~\ref{fig:sequentialBBsimple}.
In this datapath, an adder is sandwiched between three registers.
Here, the control inputs on all three registers are assumed to be
hardwired to 1, causing them to load on every positive edge.  The
inputs $A$ and $B$ to the two registers are provided by some
external agent.

\begin{figure}[ht]

\center{\scalebox{0.8}{\includegraphics{simple}}}
\caption{A simple datapath and the timing diagram describing its
behavior.}
\label{fig:sequentialBBsimple}

\end{figure}

When a signal has an unknown value, it is given a shaded regions.  Prior to
time=0, none of the registers contains a known value, therefore all
their outputs are shaded.  Since $A=3$ and $B=4$ prior to the
positive edge at time=0, the outputs of register $A/B$ equals 3/4
after the positive edge at time=0, respectively. The slight delay
in the $A_{out}$ signal becoming valid, emphasizes how the
real outputs of a register exhibit propagation delay. Note, the output of the
$B$ register is not shown because its behavior is similar to $A_{out}$ and
would clutter up the timing diagram.  Since the outputs of the $A$ and $B$
registers are unknown prior to time=0, causing the
inputs to the adders to be unknown, causing the outputs of the
adder to be unknown, causing the input to the $S$ register to be
unknown.  Since the inputs to the $S$
register are unknown when the clock edge arrives at time=0, the outputs of the
$S$ register remain unknown after that clock edge.

It might seem as if the new outputs of registers $A/B$ available
just after the clock edge at time=0 would be able to propagate
to the $S$-register's input, allowing it to latch a valid value
on the time=0 clock edge.  This is incorrect because all the
registers in the datapath latch their values at exactly the same
instant in time and then output their new values.  This assertion
is simply a restatement of the observation made on
page~\pageref{page:FFdelay}: the propagation delay of a flip flop
should be larger than the hold time in order to allow flip flops
to be daisy-chained together.

The $A$ and $B$ signals are changed at time=5, on a negative edge
of the clock, in order to keep the changes on the register inputs
as far away from a positive edge of the clock as possible.

After the rising edge of the clock at time=0, the outputs of the
$A$ and $B$ registers become valid, causing the inputs to the adder
to become valid, causing the output to become valid, causing the
input to the $S$ register to become valid.  Thus, the $S_{in}$
signal becomes valid after the positive clock edge at time=0.

When the positive clock edge at time=10 arrives at the circuit in
Figure~\ref{fig:sequentialBBsimple}, registers $A/B/S$ latch the values on their
inputs, 6/3/7, respectively.  The new outputs of $A$ and $B$ are
sent to the adder causing $S_{in} = 9$.  This value cannot be
latched into the $S$ register until the next rising edge of the
clock at time=20, because the rising edge at time=10 is long gone.

The positive clock edge at time=20 causes registers $A/B/S$ to latch 1/5/9,
respectively.  Once the analysis is started, it is a matter of waiting for a
positive clock edge, latch all the register inputs simultaneously, propagating
outputs through the combinational logic, and waiting for the next positive edge.

%%------------------------------------------------
%% Add a complex analysis problem here that uses
%% the reset and control lines.
%%------------------------------------------------

\section{Combinations}
The basic building blocks in Chapters 4 and 6 can be combined in interesting ways
to produce complex behavior.  The behavior of these digital systems can be described
using a programming-language-like syntax, initially presented in Chapter 4. First, examine
the counter, counting over a subinterval of its possible range.

\begin{verbatim}
    while(1) {
        if (count<10) count += 1;
        else count = 3;
    }
\end{verbatim}

The \verb+while(1)+ statement means that the statements contained
in \verb+while+-brackets should be executed forever.  In other
words, it is a never-ending loop.  The \verb+if+ statement checks
the count value. If it is less than 10, \verb+count+ is incremented,
otherwise the value is reset back to 3. The circuit is implemented
with a counter to perform the count-up function and a comparator
to perform the magnitude comparison between \verb+count+ and 9.
Why 9?  Because if the value is compared against 10, then the
\verb+count+ value would have to reach 10 in order for the $L$
output of the comparator to change, by which time it would be too
late to stop the \verb+count+ value from reaching 10.  The completed
circuit and a timing diagram is shown in Figure~\ref{fig:sequentialBBcomb1}.

\begin{figure}[ht]
\center{\includegraphics{comb1}}
\caption{A circuit to count between 3 and 9 and its associated timing diagram.}
\label{fig:sequentialBBcomb1}

\end{figure}

The $L$ output of the comparator is used because the condition checked is
\verb+count < 10+.  This single-bit output cannot be directly connected to the
counter's 2-bit control input because there just are not enough bits.  The
``glue" box shown in Figure~\ref{fig:sequentialBBcomb1} contains glue-logic, a
combinational logic circuit which interfaces or glues together two pieces
of logic.  When the counter's output is less than 10, $L=1$, and the
counter is supposed to count up by 1.  According to
page~\pageref{page:counter}, this requires the control to be $c_1c_0=10$.

When the counter's output is greater than or equal to than 10, $L=0$,
the counter should load 3.  According to
page~\pageref{page:counter}, a load is elicited by asserting
$c_1c_0=01$ on the counter's control input.  The
resulting truth table for the glue logic is shown in the margins.
From this table, it is easy to ascertain that $c_1 = L$ and $c_0 = L'$.

\marginpar { \tiny
$$
\begin{array}{c||c|c}
    L & c_1 & c_0    \\ \hline \hline
    0 & 0 & 1         \\ \hline
    1 & 1 & 0         \\
\end{array} $$
}

The timing diagram shows the counter starting at 6.  Since 6 is less
than 10, then $L=1$.  When $L=1$ is present on the input of the
glue logic it will output $C=c_1c_0=10$, causing the counter
to count up when the clock input arrives at time=0.  Successive clock edges
see the count value increment to 9 at time=20.  At this moment, the $L$ output
of the comparator changes to 0 causing the glue logic box to output
$C=c_1c_0=01$ telling the counter to load 3 on the next positive clock edge
at time=30.  When the value of 3 is loaded into the counter at time=30, the
$L$ output of the comparator changes to 1, causing the glue logic box to
tell the counter to count up on the next positive clock edge at time=40.
Without a finite state machine, it is difficult to get a combination of basic
building blocks to perform a sequence of actions. Difficult, but not
impossible as the following circuit shows.

Design a circuit which searches the first
100 memory locations of an eight bit wide random access memory for the smallest
value is examined.  Assume the RAM is preloaded with data values, so
only the memory needs to be read.  The approach reads each value
and checks if it is smaller than the smallest 8-bit value found
thus far.  This smallest value is stored in a register, \verb+min+, which
will be assumed as initialized to the largest possible value, 0xFF.  The ``0x"
in front of ``0xFF" is used in many program languages to signify that
``FF" is a hexadecimal number.
\label{page:minsearch}

\begin{verbatim}
1.    // assume that min is initialized to 0xFF
2.    for(i=0; i<100; i++) {
3.        MBR = RAM[i];
4.        if (MBR < min) MBR = min;
5.    }
\end{verbatim}

The complete solution is shown in Figure~\ref{fig:sequentialBBmin}.  When working through
a solution, you should work your way through the algorithm from beginning to the end
adding and connecting hardware as you proceed.  Before starting this process a few notes
are in order. The variable \verb+MBR+, standing for memory buffer register, a generic
term applied to a register used to buffer data operations between a RAM and
a datapath.  The enb and wen lines on the RAM are connected to logic levels because
we only need to read from the RAM.  This is called ``hardwiring'' the inputs.  This is
done when the input never changes, so simplifying the design.  This sort of things
occurs often enough in digital design we have a name for it - you may have to
hardwire inputs in your homework solutions.

\textbf{Line 2:} is implemented with a
counter and a comparator.  The comparator examines the counter's output, called $i$,
and stops the counter when the count value reaches 99.  The truth table for the
glue logic box is given by:

\begin{tabular}{c||c|c}
L & counter & MBR register     \\ \hline \hline
0 & count up & load         \\ \hline
1 & hold     & hold         \\
\end{tabular}

Note, you do not need to look-up the control values for the counter and
register.  It is easier to understand if you write the action that the counter
and register in the truth table. Only if you wanted to determine the logic gates
inside the glue logic block, you would need to look up the control word
values to perform these actions.

\textbf{Line 3:} is implemented with a RAM and a
register.  The address of the RAM \verb+[i]+ comes from the counter, and the
data output of the RAM is sent to the data input of a register whose output is called MBR.

\textbf{Line 4: }
is realized with a comparator and
a register.  The comparator compares the output of the MBR and the min
registers and asserts its $L$ output when MBR is less than min.  This
$L$ output runs to the control input of the min register.  The data
input of the min register comes from the data output of the MBR
register.  The control of the MBR register should stop loading when
the count value is greater or equal to 99 - this was completed in
the truth table described in Line 2.

The circuit diagram is  shown in Figure~\ref{fig:sequentialBBmin}.  Since the
problem statement did not include information about the word sizes, we have
ignored them in our design.

\begin{figure}[ht]
\center{\scalebox{0.8}{\includegraphics{min}}}
\caption{A circuit to find the smallest value in a RAM.  The min
register is initialized to 0xFF.}
\label{fig:sequentialBBmin}
\end{figure}

%%\section{Timing}
